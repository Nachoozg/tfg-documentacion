\capitulo{5}{Aspectos relevantes del desarrollo del proyecto}

En este apartado se van a presentar los aspectos más importantes de todo el desarrollo del proyecto. Se incluye desde el inicio del proyecto y la formación realizada hasta todo el proceso de desarrollo del software, el diseño y la resolución de problemas que se han ido planteando con el tiempo.

\section{Inicio del proyecto}\label{inicio-del-proyecto}

La idea de este proyecto apareció como la idea de combinar uno de los deportes que más he practicado y más disfruto viéndolo con mis estudios y la programación.

El tenis es un deporte que he practicado durante varios años de mi vida y siempre me ha gustado ver partidos de competiciones importantes para ver a los mejores en ello.

En concreto, escogí realizar una web para la gestión de una liga de tenis para los colegios de Burgos porque es algo que no existe y es novedoso. En las instalaciones de la escuela en la que iba yo a jugar a tenis había tablones en las paredes en los que se colgaban papeles con tablas de jugadores que estaban inscritos en torneos que se jugarían más adelante. Todo estaba gestionado con los tablones de anuncios y si quería consultar cuando tenía que jugar un partido o quienes estaban inscritos para jugar el torneo, tenía que acercarme hasta el lugar a consultar el papel.

Además, en el caso de el colegio al que yo iba, no había clases de tenis y por tanto, mucho menos había competición o partidos contra otras personas. 

Esa necesidad que vi, junto con el interés que me despertó el desarrollo web cuando lo descubrí, me hicieron que utilizar las herramientas que he usado para el desarrollo (Angular y .NET) me resultara más interesante y más llevadero para aprenderlo.

Una vez llegado a la idea y tras recibir el visto bueno del tutor, me puse con el desarrollo de todo el proyecto.

\section{Formación}\label{formacion}

Las herramientas que se utilizan para desarrollar este proyecto no son herramientas que se hayan visto en las asignaturas de la carrera o se ha tocado muy poco o sin relación al desarrollo web, por tanto, se requerían conocimientos de los que no disponía al comienzo del desarrollo. Para adquirir esos conocimientos que no tenía me puse a investigar en profundidad acerca de estas tecnologías y tuve que realizar cursos formativos con mucho material que me fuera útil.

Además, con el auge de la inteligencia artificial en los últimos años, finalmente surgió la idea de añadir un asistente inteligente que aparecería en forma de chat en la web y también implicaba investigación y adquirir nuevos conocimientos sobre como llevar la idea a cabo.

\subsection{Angular}

Angular era nuevo para mí, sabía que existían frameworks como React y Django pero no Angular. Realicé un par de cursos que me fueron útiles para poder aprender mucho sobre el tema:

\begin{itemize}
\tightlist
\item
  \emph{Legacy - Angular: De cero a experto} (Udemy)
  \cite{course:angular_legacy}.
\end{itemize}


.....