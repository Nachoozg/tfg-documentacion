\capitulo{6}{Trabajos relacionados}


En este apartado se hace un análisis de algunas aplicaciones web con un contenido similar o con algunas relaciones con la temática del proyecto.

La finalidad de hacer este análisis es coger ideas, características de las webs y funcionalidades que pueden resultar interesantes de implementar en el proyecto y le aportarán valor. Es por ello, que el primer paso junto con la formación en las herramientas que se utilizan para el desarrollo, ha sido este de investigación de trabajos relacionados para la obtención de ideas tanto de funcionalidad como de diseño.

\section{ATP Tour}\label{atp-tour}

ATP Tour \href{https://www.atptour.com/es}{ATP Tour en Español} es la web oficial de la asociación ATP (Asociación de Tenistas Profesionales), es la organización que regula el circuito profesional masculino de tenis. En esta web se puede ver toda la información sobre todos los jugadores y sobre los torneos que se juegan.

Algunas de las funcionalidades que me han servido de inspiración han sido:
\begin{itemize}
    \item \textbf{Calendario y resultados}: muestra un calendario en forma de lista en el que aparece por meses que torneos se juegan.
    \item \textbf{Clasificación general}: tiene una tabla de la clasificación general de cada jugador en el ranking global.
    \item \textbf{Estadísticas y fichas de jugadores}: muestra estadísticas e información interesante sobre cada jugador inscrito.
\end{itemize}

Al ser una web oficial y enfocada en la competición profesional, es la que más funcionalidades y características interesantes tiene respecto al resto de webs que puede haber.

\imagen{ficha-jugador.png}{Vista de la ficha de los jugadores}{0.8}

\section{Liga de tenis Madrid}\label{tenis-madrid}

La web de la liga de tenis de Madrid \href{https://www.rankingmadridcapital.com/#body}{Liga de tenis Madrid} es una web que ofrece un espacio en el que los jugadores de cualquier nivel pueden inscribirse y competir.

En el caso de esta web, me aportó la idea de meter un mapa interactivo en el que poder poner punteros para indicar los lugares en los que se puede jugar en la ciudad. El resto del material de la web no me ha sido de mucha utilidad.


\imagen{mapas-web.png}{ Vista del mapa interactivo}{0.8}

\section{University Tennis League}\label{uni-tennis-league}

Esta web \href{https://www.university-tennis.com/teams}{University Tennis League} es una web para gestionar una liga de tenis en diferentes universidades de fuera de España.

Me sirvió para coger ideas sobre apartados a incluir en mi proyecto, porque el contenido general es amplio, pero no demasiado detallado. Por otro lado, el diseño no es demasiado amigable y es por ello que solo me fijé en el contenido, en el que incluye:

\begin{itemize}
    \item \textbf{Información sobre los equipos}: muestra una lista de los equipos inscritos en la competición con algunos datos informativos de cada uno de ellos.
    \item \textbf{Resultado de partidos}: tiene un apartado de la web en el que muestra los partidos que se han jugado, cuando se han jugado y los resultados de ellos.
    \item \textbf{Clasificación}: muestra una tabla con la clasificación de los equipos, que aunque no es interactiva por ser una foto, muestra la información que hace a los equipos llegar a ese puesto en la clasificación.
\end{itemize}

\imagen{partidos-web.png}{ Vista de la lista de partidos jugados}{1.0}


\section{College Tennis League 2024-25}\label{college-tennis-league}

La web \href{https://playwaze.com/cambridge-university-tennis-24-25/uszhp193kzm8l/league-display/leagues/jdt6ycvecilz4}{College Tennis League 2024-25} permite gestionar toda la información de una liga de tenis de colegios.

Las funcionalidades o características interesantes que tiene son:
\begin{itemize}
    \item \textbf{Clasificación}: muestra la tabla de clasificación con un buen diseño y datos importantes.
    \item \textbf{Resultados de los partidos}: la información de los partidos jugados o por jugarse también tiene un buen diseño y muestra la información necesaria.
    \item \textbf{Estadísticas}: tiene una ventana que debería mostrar estadísticas pero para todos los campos pone que no tiene ningún dato disponible, por tanto no son de ninguna utilidad.
\end{itemize}

\imagen{college-league.png}{ Vista de la tabla de clasificación}{1.0}