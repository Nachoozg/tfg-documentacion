\capitulo{4}{Técnicas y herramientas}

\section{Angular}\label{angular}
Angular es un framework de desarrollo de aplicaciones web que es de código abierto y está mantenido por Google, está basado en TypeScript, que es similar a JavaScript y aporta un tipado estático y unas características avanzadas de ES6+, que mejoran la escalabilidad y la capacidad de mantenimiento de los proyectos. 

El principal núcleo que diferencia y da ventaja a Angular respecto a otros frameworks es el concepto de componentes, que son unidades que combinan html, typescript y css. Cada componente representa una de las partes que se ven en la interfaz de usuario y se pueden anidar unos con otros formando jerarquías que definirán la estructura final que tendrá la aplicación. Estos componentes se agrupan en modulos, que permiten organizar el código en diferentes funcionalidades y admitir de esa forma la carga perezosa (lazy loading) para que se cargue solo la informacion esencial en cada ventana y se mejore el rendimiento.

Para hacer la gestión de la comunicación entre componentes y servicios, Angular tiene un sistema que se llama inyección de depencencias, con el que se registran en los "injectors" o contenedores los servicios que sirven para acceder a datos.

Angular tambien incluye un enrutador (Router) que permite definir rutas y navegacion dentro de la aplicacion web, gestionando los parámetros, posibles rutas hijas y los guards que controlan el acceso a apartados de la aplicacion segun los permisos que se quieran dar. Junto con angular cli, ofrece un ecosistema muy maduro. lo que lo convierte en una opción sólida para construir aplicaciones web robustas, mantenibles y de gran escala.

\section{Angular CLI}\label{angular-cli}
Es una herramienta de línea de comandos que acelera la creación, compilación y el despliegue de los proyectos, de manera que permite aplicar mejores prácticas de forma automatica.

AÑADIR MÁS...