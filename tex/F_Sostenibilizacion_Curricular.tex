\apendice{Anexo de sostenibilización curricular}

\section{Introducción}

En este anexo se añaden las reflexiones personales sobre los aspectos relacionados con la sostenibilidad que se han abordado con el desarrollo de este trabajo. Todo con el objetivo final de mostrar cómo se han conseguido y aplicado las competencias de sostenibilidad durante la realización del Trabajo de Fin de Grado. En el documento \href{https://www.crue.org/wp-content/uploads/2020/02/Directrices_Sosteniblidad_Crue2012.pdf}{CRUE} se definen las competencias de sostenibilidad.

\section{Competencias de sostenibilidad}

\textbf{Conciencia y comprensión de la sostenibilidad}

Durante el desarrollo del proyecto, se ha adquirido una mayor concienciación sobre lo importante que es la sostenibilidad en el ámbito tecnológico y se ha comprendido cómo la tecnología desarrollada puede influir en la educación y en la sociedad. Tener en cuenta que las decisiones pueden tener un gran impacto en el medio ambiente y en la sociedad es esencial.

Por ejemplo, eligiendo servicios para alojamiento, se ha seleccionado AWS (Amazon Web Services) entre otras posibilidades similares por su comprensión sobre este tema. Se puede ver la información que proporcionan en la web de \href{https://aws.amazon.com/es/sustainability/}{Computación en la nube sostenible | Amazon Web Services}

\newpage

\textbf{Impacto social y accesibilidad del proyecto}

El tema de la sostenibilidad también se aplica para considerar el impacto social de la tecnología. Desarrollando este proyecto se ha trabajado con cuidado para asegurar que la aplicación web es accesible por usuarios con diferentes niveles de habilidades, capacidades y conocimientos. Se ha hecho un diseño amigable y un flujo sencillo de utilización para no causar problemas con su uso.

\textbf{Utilización sostenible de recursos}

El diseño y el flujo de trabajo del proyecto se han llevado a cabo teniendo un enfoque en la eficiencia de los recursos tecnológicos. Se han intentado hacer el mínimo de peticiones a APIs externas que consumen recursos. Se ha priorizado el uso de software libre y proyectos de código abierto frente a herramientas propietarias, fomentando de esa manera un entorno de desarrollo más sostenible.

\textbf{Educación y sensibilización}

El desarrollo de este proyecto ha sido también una muy buena oportunidad para enseñar y sensibilizar sobre la sostenibilidad en el ámbito de la tecnología. Al tener compañeros con los que compartir dificultades encontradas y desafíos enfrentados con el proceso de desarrollo, se ha ayudado a concienciar sobre la importancia de la sostenibilidad en la tecnología. Con la documentación del proyecto también se ha pretendido fomentar la colaboración y el aprendizaje colectivo.

\section{Conclusión}

Al integrar los servicios de sostenibilidad durante el desarrollo del proyecto, se han adquirido conocimientos muy importantes y educativos, de forma que me han permitido valorar lo importante que es tener en cuenta el impacto ambiental y social que tienen las decisiones que se toman.

Habiendo optimizado recursos, elegido herramientas amigables con el medio ambiente y el impacto social o herramientas y productos libres frente a propietarios, he aprendido que existe la posibilidad de crear proyectos tecnológicos que sean modernos y tengan responsabilidad.

Con este tipo de desarrollos se beneficia tanto al medio ambiente, como a la sociedad, como se añade mucho valor al proyecto haciendo que sea más completo y amigable con lo que se espera actualmente sobre una solución de este tipo.

Por tanto, las competencias que se han adquirido durante el trabajo realizado son muy favorables porque han hecho que el proyecto se complete con éxito, pero también porque aportan una muy buena base para los futuros proyectos que se lleven a cabo ya sea propios o más profesionales. Estar comprometidos con la sostenibilidad nos permite construir un futuro mejor y amigable para todo el mundo.