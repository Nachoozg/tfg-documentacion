\apendice{Plan de Proyecto Software}

\section{Introducción}

En esta fase, se ha realizado la planificación del proyecto, que es un punto muy importante para todos los proyectos. Lo que se hace es la planificación del tiempo, trabajo y dinero que supodrá la realización del proyecto y para conseguirlo, se hace un análisis de cada parte.

El objetivo de este apartado es permitir hacer una vista general con detalles sobre la gestión del proyecto que estamos realizando, de manera que se indica como se ha hecho la planificación, la ejecución y el desarrollo de principio a fin.

La fase de la planificación se divide en dos partes:
\begin{itemize}
\tightlist
\item
  \textbf{Planificación temporal}: En esta parte se hace un desglose de los tiempos que se han empleado para hacer cada parte del proyecto. Para ello, también hay una fecha de inicio y una fecha en la que está previsto terminar.
\item
  \textbf{Estudio de viabilidad}: Esta parte es para ver la viabilidad del desarrollo del proyecto. Dentro de este apartado, se puede dividir en dos apartados, que son la viabilidad económica (estimación de los costes de la realización del proyecto) y la viabilidad legal (análisis de las regulaciones que tiene el desarrollo del proyecto como licencias, uso de software y leyes).
\end{itemize}

\section{Planificación temporal}

La planificación que se ha seguido ha sido ir haciendo el desarrollo de forma continua, haciendo reuniones o entregas cada cierto tiempo para ver los avances del proyecto y ver que se tiene que incluir y modificar para la siguiente entrega.

Las entregas y reuniones al inicio del desarrollo se hicieron cuando el progreso era notable respecto a la anterior entrega, ya que en un principio era cuando más costaba avanzar porque es cuando surgen la mayoría de problemas y se tiene menos experiencia con el uso de las herramientas de desarrollo. A medida que el proyecto cogía forma, las entregas pasaron a ser cada menos tiempo, hasta que en los últimos compases pasaron a realizarse cada semana.

A continuación, se muestra un desglose de las entregas que se han ido realizando a lo largo de todo el desarrollo, junto con una descripción sobre las tareas realizadas en cada una de ellas.

\subsection{Entrega 1 (01/02/25 - 10/02/25)}

Con esta primera entrega se estableció el inicio del proyecto. En ella se decidió el tema sobre el que se iba a realizar el proyecto, la metodología que se utilizaría y se instalaron y configuraron todas las herramientas y software necesarias para comenzar con el desarrollo, tanto para la gestión del proyecto (GitHub) como para el desarrollo del proyecto (Angular y .NET).

Además de comenzar con toda la fase de documentación y formación relacionada con el proyecto para obtener conocimientos y poder empezar con el desarrollo de la aplicación.

\subsection{Entrega 2 (11/02/25 - 19/03/25)}

Los objetivos de esta nueva entrega eran: aprender a utilizar las herramientas de desarrollo (Angular, .NET y HeidiSQL), realizar un curso online que permitiera aprender a utilizar Angular con fluidez, otro curso para hacer lo mismo con .NET, sobretodo enfocado en el desarrollo de APIs y investigar sobre el uso de HeidiSQL para realizar bases de datos con MySQL. En los cursos, también se comenzaba con la instalación correcta de cada componente que haría falta para el desarrollo.

Además, durante el tiempo que se estuvieran haciendo los cursos y obteniendo formación, se comenzaría con el desarrollo de una estructura básica del proyecto, que estaría compuesta por el front-end en Angular con la estructura a partir de la que se pudieran implementar mejoras, la creación de una base de datos inicial con las tablas necesarias para comenzar y la conexión de esa base de datos con el back-end en .NET con los primeros endpoints para tener la API montada para gestionar esos datos.

Para saber que apartados se podían incluir en el proyecto, también en esta fase se realizaría una investigación sobre páginas con objetivos y temáticas parecidas para sacar conclusiones y observar el diseño.

Las horas dedicadas durante el periodo de tiempo de esta tarea se desglosan de la forma que se ve en la tabla, teniendo en cuenta que los cursos son muy completos y tienen una duración de muchas horas, y por tanto, mucho tiempo invertido diario para comenzar cuanto antes con el desarrollo.

\begin{longtable}{@{}lll@{}}
  \toprule
  \rowcolor{gray!20}
  Tarea & Etiqueta & Número de horas \\ 
  \midrule
  \endhead
  Curso de Angular  & Formación  & 2 horas y media diarias \\ 
  \midrule

  Curso de .NET  &  Formación    &   1 hora diaria \\ 
  \midrule

  Desarrollo del proyecto  &  Programación  & 1 hora diaria \\ 
  \midrule

  Aprendizaje de uso de HeidiSQL  &  Investigación  &  2 horas \\ 
  \midrule

  Investigación contenido a incluir  &  Investigación  &  6 horas \\ 
  \midrule

  \bottomrule
  \caption{Desglose de horas -- Entrega 2.}
\end{longtable}


\subsection{Entrega 3 (20/03/25 - 09/04/25)}

Los objetivos de esta entrega fueron: terminar la formación que se estaba haciendo con los cursos, arreglar errores de la anterior entrega, añadir nuevos apartados en la aplicación web como la posibilidad de almacenar resultados y la tabla de clasificación. Además, se comenzó con la documentación del proyecto, en específico con el apartado de la memoria.

\begin{longtable}{@{}lll@{}}
  \toprule
  \rowcolor{gray!20}
  Tarea & Etiqueta & Número de horas \\ 
  \midrule
  \endhead
  Curso de Angular  & Formación  & Media hora diaria \\ 
  \midrule

  Desarrollo del proyecto  &  Programación  &  2 horas diarias \\ 
  \midrule

  Documentación  &  Documentación  &  3 horas \\ 
  \midrule

  \bottomrule
  \caption{Desglose de horas -- Entrega 3.}
\end{longtable}


\subsection{Entrega 4 (10/04/25 - 20/04/25)}

En esta entrega los objetivos fueron: añadir funcionalidades como la subida y visualización de imágenes, creación de tablas dinámicas, que cambiaran en tiempo real al modificar datos, cálculo de puntos de cada colegio de forma dinámica y conteo de partidos jugados. Actualización de controladores, creación de nuevos DTO (Data Transfer Object) en .NET para hacer cálculos más complejos y más rápidos en las tablas. 

Algunas de las partes del desarrollo de esta entrega llevaron muchas horas de programación por errores que surgieron y se tardó más tiempo de lo que estaba previsto.

\begin{longtable}{@{}lll@{}}
  \toprule
  \rowcolor{gray!20}
  Tarea & Etiqueta & Número de horas \\ 
  \midrule
  \endhead
  Subida y visualización de imágenes  & Programación  & 14 horas \\ 
  \midrule

  Creación de tablas dinámicas  &  Programación  &  11 horas \\ 
  \midrule

  \bottomrule
  \caption{Desglose de horas -- Entrega 4.}
\end{longtable}

\subsection{Entrega 5 (21/04/25 - 07/05/25)}

En esta entrega los objetivos fueron: Añadir la funcionalidad de marcar la ubicación en el formulario de partidos, incluyendo un mapa interactivo para seleccionar coordenadas, añadir ventanas de login y de registro que permitieran crear y validar usuarios con su rol específico y restringir acceso a apartados de la web y a datos a cada rol específico. Además de arreglar algunos fallos que había en la tabla de clasificación y añadir mejoras de visualización y funcionalidad en ella. También se acordó seguir escribiendo la documentación de la memoria.

Tareas como la integración de mapas interactivos en la web y la validación por roles, también implican una formación sobre como realizar su desarrollo y en este caso algo más, ya que se tuvo que rehacer el código del componente de los mapas porque se decidió cambiar de una API de pago a una gratuita.

\begin{longtable}{@{} p{6cm} p{3cm} p{3cm} @{}}
  \toprule
  \rowcolor{gray!20}
  Tarea & Etiqueta & Número de horas \\ 
  \midrule
  \endhead
  Funcionalidad ubicación  & Programación  & 10 horas \\ 
  \midrule

  Ventanas de login y registro con validación de roles  &  Programación  &  17 horas \\ 
  \midrule

  Mejoras y arreglos tabla clasificación  & Programación  & 8 horas \\ 
  \midrule

  Documentación de la memoria & Documentación  & 6 horas \\ 
  \midrule

  \bottomrule
  \caption{Desglose de horas -- Entrega 5.}
\end{longtable}


\subsection{Entrega 6 (08/05/25 - 14/05/25)}

Los objetivos de esta entrega fueron: mejorar la interfaz de usuario para hacer la web más sencilla de utilizar y que su diseño fuera más atractivo. Actualización de las secciones de inicio, login y registro añadiendo nuevos botones y estilos, además de añadir imágenes para mejorar el diseño. Rediseño de las ventanas de listas (tanto de jugadores, como de colegios y partidos), añadir modales para confirmar cambios importantes en los datos, formularios adaptados al diseño general, lista de clasificación adaptada y ventanas de vista de información específica de cada jugador y colegio.

Además, en esta entrega se añadía una nueva funcionalidad que era un chatbot inteligente con inteligencia artificial integrada, diseñado para responder a preguntas sobre la información de la web. Incluir este chatbot implicó más horas de programación porque se necesitaba formación para hacerlo y tiempo para desarrollarlo y resolver errores, además implicaba hacer la parte visual del chat y toda la lógica en el back-end. La ventana del chat se creó directamente adaptada al diseño que seguía toda la web.

También se siguió escribiendo la documentación.

\begin{longtable}{@{} p{6cm} p{3cm} p{3cm} @{}}
  \toprule
  \rowcolor{gray!20}
  Tarea & Etiqueta & Número de horas \\ 
  \midrule
  \endhead
  Rediseño de interfaz  & Programación  & 12 horas \\ 
  \midrule

  Chatbot IA  &  Programación  &  17 horas \\ 
  \midrule

  Documentación de la memoria & Documentación  & 4 horas \\ 
  \midrule

  \bottomrule
  \caption{Desglose de horas -- Entrega 6.}
\end{longtable}

\subsection{Entrega 7 (15/05/25 - 21/05/25)}

Los objetivos de esta entrega fueron: asociar partidos a cada jugador de cada colegio y añadir ventana de estadísticas de cada jugador. Mejora del funcionamiento del chatbot para adaptarse a más preguntas y responder sin fallos. Además, seguir escribiendo para dejar poco por hacer en la documentación de la memoria y redactar los anexos.

\begin{longtable}{@{} p{6cm} p{3cm} p{3cm} @{}}
  \toprule
  \rowcolor{gray!20}
  Tarea & Etiqueta & Número de horas \\ 
  \midrule
  \endhead
  Asociación de jugadores a partidos  & Programación  & 4 horas \\ 
  \midrule

  Mejora del funcionamiento del chatbot  & Programación  & 5 horas \\ 
  \midrule

  Documentación de la memoria & Documentación  &  10 horas \\ 
  \midrule

  \bottomrule
  \caption{Desglose de horas -- Entrega 7.}
\end{longtable}


\subsection{Entrega 8 (22/05/25 - 27/05/25)}

Los objetivos de esta entrega fueron: mejora del diseño, adaptándolo a pantallas pequeñas como las de dispositivos móviles, arreglo de errores con los mapas interactivos, arreglo de errores de visualización de mensajes en el chatbot y mejora del reconocimiento del bot para responder a más preguntas y de una manera mejor. Finalizar la documentación de la memoria y seguir con los anexos.

\begin{longtable}{@{} p{6cm} p{3cm} p{3cm} @{}}
  \toprule
  \rowcolor{gray!20}
  Tarea & Etiqueta & Número de horas \\ 
  \midrule
  \endhead
  Adaptación y mejora del diseño  & Programación  & 9 horas \\ 
  \midrule

  Mejora del funcionamiento del chatbot  & Programación  & 2 horas \\ 
  \midrule

  Documentación de la memoria & Documentación  &  10 horas \\ 
  \midrule

  \bottomrule
  \caption{Desglose de horas -- Entrega 8.}
\end{longtable}

\subsection{Entrega 9 (28/05/25 - 06/06/25)}

Los objetivos de esta entrega fueron: adaptación del código para subirlo a una máquina virtual de AWS (Amazon Web Services) y subida del proyecto a la máquina. Revisar fallos de la memoria y terminar los anexos.

La subida del proyecto implicó bastante investigación y mucho tiempo de arreglo de errores y búsqueda de soluciones.

\begin{longtable}{@{} p{6cm} p{3cm} p{3cm} @{}}
  \toprule
  \rowcolor{gray!20}
  Tarea & Etiqueta & Número de horas \\ 
  \midrule
  \endhead
  Adaptación del código  & Programación  & 1 hora \\ 
  \midrule

  Subida del proyecto a AWS  & Investigación y gestión & 20 horas \\ 
  \midrule

  Documentación de la memoria & Documentación  &  25 horas \\ 
  \midrule

  \bottomrule
  \caption{Desglose de horas -- Entrega 9.}
\end{longtable}

En la figura de debajo se puede ver una gráfica que genera GitHub de forma automática con las aportaciones que se han ido haciendo a los diferentes repositorios, en este caso, los repositorios de código tanto de Angular como de .NET y de la documentación. Como se puede ver, la aportación ha sido separada en el tiempo de forma incremental.

\imagen{grafico-github.png}{Gráfico de GitHub con las aportaciones en el tiempo}{0.6}

\section{Estudio de viabilidad}

En esta sección se hace un estudio detallado de la viabilidad económica y legal para garantizar que el proyecto sea sostenible a largo plazo.

\subsection{Viabilidad económica}

Se analizarán los costes y beneficios que podría suponer el proyecto si se hubiese desplegado en el entorno profesional o empresarial.

\subsubsection{Costes}


\textbf{Costes de personal:}
El proyecto ha sido desarrollado por un único empleado y se estima que se han realizado unas 380 horas de trabajo totales repartidas en 4 meses. Esto implica que la carga de trabajo es de unas 22 horas a la semana. El salario del alumno se puede estimar como 15€/hora, lo que hace que el cálculo sea:

\begin{longtable}{@{} p{8cm} p{3cm} @{}}
  \toprule
  \rowcolor{gray!20}
  Concepto & Coste \\ 
  \midrule
  
  22 h/semana * 15€/h * 4 semanas/mes  & 1320€ \\ 
  \midrule
  
  \endhead
  Salario mensual bruto  & 1320€ \\ 
  \midrule

  Retención IRPF (15\%) & 198€ \\ 
  \midrule

  Seguridad Social (30\%) & 83,82€ \\ 
  \midrule

  Salario mensual neto & 1038,18€ \\ 
  \midrule
  \midrule

  Total 4 meses neto & 4152,72€ \\ 
  \midrule

  Total 4 meses bruto & 5280€ \\ 
  
  \bottomrule
  \caption{Costes de personal.}
\end{longtable}

Los impuestos que se han tenido en cuenta son:
\begin{itemize}
\item Contingencias comunes: 4,70\%
\item Desempleo: 1,55\%
\item Fondo de Garantía Salarial (FOGASA): 0,20\%
\item Formación profesional: 0,10\%
\end{itemize}

El total sale a 6,35\% del bruto de aportación del trabajador. Además, la empresa paga alrededor del 30\%.

Para hacer todos estos cálculos se han consultado diferentes páginas oficiales con información actualizada a 2025. \cite{web:retencion-bbva} \cite{web:tipos-cotizacion} \cite{web:retencion-irpf}.

\imagen{tipos-cotizacion.png}{Régimen general de la Seguridad Social}{1.2}

Por tanto, teniendo en cuenta impuestos, lo que la empresa tiene que pagar por el empleado son 1320€ al mes, lo que supone 5280€ en los 4 meses de trabajo.

Además, también se cuenta con un profesor guiando y ayudando al alumno, lo que supone un salario de alrededor de 40€/hora. El tutor trabaja más o menos unas 2 horas por cada dos semanas, 1 hora a la semana. Por lo cual, los cálculos salen:

\begin{longtable}{@{} p{8cm} p{3cm} @{}}
  \toprule
  \rowcolor{gray!20}
  Concepto & Coste \\ 
  \midrule
  
  1 h/semana * 40€/h * 4 semanas/mes  & 160€ \\ 
  \midrule
  
  \endhead
  Salario mensual neto  & 160€ \\ 
  \midrule

  Salario mensual bruto & 171,12€ \\ 
  \midrule
  
  \bottomrule
  \caption{Costes de profesor.}
\end{longtable}

Con todo esto calculado, obtenemos que la empresa debería pagar tanto al desarrollador como al profesor, por lo tanto esto generaría un gasto mensual al mes de 1491,12€. 

Como el proyecto ha durado 4 meses, el coste total sería de unos 5964,48€.

\textbf{Costes de hardware:}

Los recursos hardware que se han utilizado para el desarrollo del proyecto han sido únicamente un ordenador portátil cuyo coste fue de 450€. Este portátil ha tenido uso durante los 4 años de la carrera, por lo que se puede decir que ha sido bastante amortizado, aunque si contamos solamente el desarrollo del proyecto, han sido 4 meses. Vamos a poner que la amortización completa está en los 5 años.

\begin{longtable}{@{} p{4cm} p{1cm} p{3cm} p{3cm} @{}}
  \toprule
  \rowcolor{gray!20}
  Concepto & Coste & Coste amortizado durante el proyecto & Coste amortizado total\\ 
  \midrule
  \endhead
  
  Ordenador portátil  & 450€ & 30€ & 360€\\ 
  \midrule

  
  \bottomrule
  \caption{Costes de hardware.}
\end{longtable}


\textbf{Costes de software:}

Para realizar el proyecto todas las herramientas que se han utilizado han sido gratuitas, con una excepción que es el sistema operativo del ordenador. La amortización completa del software podemos establecerla en 2 años.

Para que la aplicación web funcionara mas fluida y tuviera elementos con más funcionalidades y más sencillos de implementar, se podría cambiar el plan gratuito de la máquina de AWS por uno de pago y añadirla características más potentes (más núcleos, más memoria, etc.). Por otro lado, también se podrían pagar licencias de APIs como la de Leaflet para integrar los mapas, por la de Google Maps o alguna similar. También se podría pagar por el uso de otro modelo de Gemini más potente para el chatbot y que de esa manera fuera algo más inteligente. 

\begin{longtable}{@{} p{4cm} p{1cm} p{3cm} p{3cm} @{}}
  \toprule
  \rowcolor{gray!20}
  Concepto & Coste & Coste amortizado durante el proyecto & Coste amortizado total\\ 
  \midrule
  \endhead
  
  Windows 11 Pro  & 259€ & 43,16€ & 259€\\ 
  \midrule

  
  \bottomrule
  \caption{Costes de software.}
\end{longtable}

\newpage

\textbf{Coste Total:}

Para calcular todos los costes de todos los aspectos del proyecto, hay que tener en cuenta otros costes extras como la tarifa de la luz, la del internet, la del lugar en el que se desarrolla el proyecto, etc.. Para hacer una estimación, se calculará como el 15\% de la suma de los gastos.

Con todo esto calculado, el coste total es:

\begin{longtable}{@{} p{6cm} p{3cm} @{}}
  \toprule
  \rowcolor{gray!20}
  Concepto & Coste total \\ 
  \midrule
  \endhead
  Coste de personal  & 5964,48€ \\ 
  \midrule

  Coste de hardware  & 420€ \\ 
  \midrule

  Coste de software & 215,84€ \\ 
  \midrule

  Coste indirecto & 989,60€ \\ 
  \midrule
  \midrule

  \rowcolor{blue!20}
  Total & 7589,92€ \\

  \bottomrule
  \caption{Costes totales.}
\end{longtable}


\textbf{Beneficios:}

Este proyecto se ha desarrollado con fines educativos, por tanto, no se obtendrá ningún beneficio con el uso de la aplicación.

En caso de querer obtener beneficio, se podría buscar la manera de incorporar anuncios en la página y así recibir parte del dinero por cada anuncio visto. También se podría cobrar a los colegios y a los árbitros una suscripción por usar las ventajas que tiene su rol y estar inscritos en la liga.

\newpage

\subsection{Viabilidad legal}

En esta sección se va a realizar un estudio de todo lo relacionado con las licencias que tienen las librerías y las herramientas que se han utilizado para desarrollar el proyecto. 

Se han utilizado las siguientes herramientas y librerías:

\begin{longtable}{@{} p{5cm} p{3cm} p{5cm} @{}}
  \toprule
  \rowcolor{gray!20}
  Herramienta/Librería & Versión & Licencia \\ 
  \midrule
  \endhead
  
  Angular  & 19.1.7 & MIT \\ 
  \midrule
  
  Angular CLI  & 19.1.8 & MIT \\ 
  \midrule

  npm  & 10.9.2 & MIT \\
  \midrule

  Node  & 22.14.0 & MIT \\
  \midrule

  rxjs  & 7.8.2 & Apache 2 \\
  \midrule

  TypeScript  & 5.7.3 & Apache 2 \\
  \midrule

  zone.js  & 0.15.0 & MIT \\
  \midrule

  Bootstrap  & 5.3.5 & MIT \\
  \midrule

  @angular/material  & 19.2.8 & MIT \\
  \midrule

  @angular/animations  & 19.1.0 & MIT \\
  \midrule

  @angular/router  & 19.1.0 & MIT \\
  \midrule

  @fullcalendar/angular  & 6.1.16 & MIT \\
  \midrule

  leaflet  & 1.9.4 & BSD \\
  \midrule

  ngx-toastr  & 19.0.0 & MIT \\
  \midrule

  EntityFrameworkCore.Tools  & 8.0.2 & MICROSOFT \\
  \midrule

  Mscc.GenerativeAI  & 0.9.0 & Google AI Studio \\
  \midrule

  Mscc.GenerativeAI.Web  & 0.9.0 & Google AI Studio \\
  \midrule

  Pomelo.EFC.MySql  & 8.0.2 & MIT \\
  \midrule

  Swashbuckle.AspNetCore  & 6.6.2 & MICROSOFT \\
  \midrule
  
  \bottomrule
  \caption{Licencias y Versiones.}
\end{longtable}

Todas las librerías y herramientas que se han utilizado con el desarrollo del proyecto tienen licencias que son de software libre, con lo que no existe ninguna limitación de uso con la que se tenga que aplicar una licencia más restringida en el proyecto.

La licencia MIT permite el uso de forma libre siempre y cuando se proporciona la fuente de los autores originales y se les mencione en la documentación del proyecto.