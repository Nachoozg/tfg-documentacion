\apendice{Documentación de usuario}

\section{Introducción}
En este apéndice se explican todos los aspectos que tienen relación con la aplicación por la parte del usuario. Se muestran los requisitos de usuario y el manual que indica como se usa la aplicación.

\section{Requisitos de usuarios}
La aplicación de la liga de tenis de colegios de Burgos está desplegada y es accesible por todo el mundo, lo que hace que los requisitos no sean demasiado exigentes. 

Solo se necesita tener acceso a internet con un dispositivo, ya sea un ordenador, una tablet o un móvil, y que el dispositivo tenga instalado un navegador web que tenga dentro de lo posible estas versiones:

\begin{itemize}
    \item 
    \textbf{Chrome:} versión 132 en adelante
    \item 
    \textbf{Brave:} versión 1.73.5 en adelante
    \item 
    \textbf{Opera:} versión 83.5 en adelante
    \item 
    \textbf{Mozilla Firefox:} versión 120.0 en adelante
    \item 
    \textbf{Microsoft Edge:} versión 131.0 en adelante
    \item 
    \textbf{Safari:} versión 17.1 en adelante
\end{itemize}


\section{Instalación}
Como la aplicación está desplegada para que cualquier usuario pueda acceder a ella, no es necesario instalar nada, solamente hay que acceder al enlace de la web desde un navegador. \href{http://ligatenisburgos.es/}{Liga de tenis de colegios de Burgos}



\section{Manual del usuario}

En este manual se proporciona una descripción detallada de como se utiliza la aplicación con el objetivo final de hacer que resulte más sencillo para el usuario y sea más comprensible el flujo de funcionamiento.

El primer paso y fundamental es acceder a la web a través del enlace \href{http://ligatenisburgos.es/}{Liga de tenis de colegios de Burgos}.

\textbf{Inicio}

La primera pantalla que se ve al acceder a la web es la de inicio, tiene una barra superior de navegación y la página principal en sí, que contiene algo de información y el calendario de partidos.

\imagen{a0.png}{Pantalla de inicio}{0.59}

\textbf{Barra de navegación}

La barra de navegación muestra el logotipo de la aplicación, junto con el botón de inicio para volver a la página de inicio si no estás en ella, tiene también los botones que redirigen a otras ventanas diferentes y el botón de inicio de sesión.

\imagen{a.png}{Barra de navegación completa}{1}

Esta barra de navegación se puede encontrar en todas las ventanas de la aplicación y desde ella se puede acceder a todas las funcionalidades en el momento que se quiera. Al ser tan importante este elemento, también se ha adaptado para utilizarla en pantallas más pequeñas.

\imagen{a1.png}{Barra de navegación en vista de móvil}{1}

\imagen{a2.png}{Barra de navegación expandida en vista de móvil}{1}

En el caso del botón de iniciar sesión, cuando se tiene una sesión iniciada cambia y muestra el nombre de usuario.

\imagen{a3.png}{Cambio en el botón de inicio de sesión}{0.3}

Al pinchar sobre ese nombre de usuario, se despliega un botón de cierre de sesión que abre una ventana para confirmar el cierre para asegurarse de que no se ha pinchado sin querer.

\imagen{a4.png}{Botón de cierre de sesión}{0.3}

\imagen{a5.png}{Confirmación de cierre de sesión}{0.5}

\textbf{Página principal}

La ventana principal es la pantalla de inicio, en la que se muestran la información de bienvenida a la web y el calendario con los partidos que se han jugado o están por jugarse.

\imagen{a6.png}{Cuadro de bienvenida}{0.8}

\imagen{a7.png}{Calendario de partidos}{0.9}

\textbf{Inicio de sesión}

Al pinchar en el botón de inicio de sesión, se abre la ventana correspondiente, donde hay que rellenar el formulario con los datos correctos del usuario asignado.

\imagen{loginFinalVista.png}{Ventana de inicio de sesión}{0.7}

Cuando iniciamos sesión correctamente, se muestra esta tarjeta que lo confirma y asegura al usuario que ha entrado en su sesión.

\imagen{a8.png}{Tarjeta confirmación}{0.3}

En la ventana del inicio de sesión también se muestra un enlace al registro para que si el usuario no tiene cuenta asociada, pueda crear una.

\imagen{a9.png}{Botón de acceso a registro}{0.3}

\newpage

\textbf{Registro de usuario}

Para el registro, se le muestra al usuario un formulario en el que introducir sus datos y se le pide elegir un rol entre colegio y árbitro.

\imagen{registro.png}{Ventana de registro de usuario}{0.7}

También en este caso se muestran tarjetas de aviso al usuario en caso de que los datos que introduzca no sean correctos para guiarle a conseguir su nuevo usuario con credenciales correctas.

\imagen{a10.png}{Tarjeta de aviso contraseña no válida}{0.5}

\imagen{a11.png}{Tarjeta de aviso contraseñas diferentes}{0.5}

\imagen{a12.png}{Tarjeta de aviso correo ya registrado}{0.5}

\textbf{Ventana de clasificación}

Si se accede desde la barra de navegación a la ventana de clasificación, se muestra una tabla con la clasificación de cada colegio, la cual se puede expandir para cada colegio y se verán los resultados de los últimos partidos que ha jugado.

\imagen{clasificacionComprimida.png}{Tabla de clasificación comprimida}{1}

\imagen{clasificacionExtendida.png}{Tabla de clasificación expandida}{1}

\textbf{Ventana de colegios}

En la ventana de colegios se muestra a primera vista un listado de los colegios que hay inscritos en la liga.

\imagen{listadoFinalVista.png}{Listado de colegios desde la vista de un usuario común}{1}

\imagen{listadoFinalVistaAdmin.png}{Listado de colegios desde la vista de un usuario con roles específicos}{1}

Si se pincha en el nombre del colegio, muestra la vista en detalle de ese colegio con su nombre, imagen y número de jugadores.

\imagen{verColeFinal.png}{Interfaz de la vista en detalle de los colegios}{1}

En el caso de tener roles privilegiados, se pueden agregar y editar tanto los colegios, como los jugadores y los partidos. La interfaz es muy similar entre ellos.

\imagen{agregarCFinal.png}{Interfaz que tienen los formularios de agregar y editar (Colegio, Jugador y Partido)}{1}

\textbf{Ventana de jugadores}

Cuando se pincha en la barra de navegación sobre jugadores, se muestra la lista de todos los jugadores que tenemos inscritos en la liga.

\imagen{b1.png}{Listado de los jugadores inscritos en la liga}{1}

\newpage

\textbf{Ventana de partidos}

Cada jugador también tiene su vista en detalle al pinchar sobre su nombre en el listado.

\imagen{verJugaFinal.png}{Interfaz de la vista en detalle de los jugadores}{1}

Al pinchar en la barra de navegación sobre los partidos, se muestra el listado de los partidos que se juegan durante el transcurso de la liga.

\imagen{b2.png}{Listado de partidos que se juegan en la liga}{1}

Cada partido también muestra información específica cuando pichamos sobre la fecha en la que se juega el partido.

\imagen{verPartiFinal.png}{Interfaz de la vista en detalle de los partidos}{1}

Al pinchar en el botón de "Ver dirección", se abre una ventana con el mapa interactivo que permite ver donde se juega el partido.

\imagen{modalMapafinal.png}{Interfaz de la vista del mapa interactivo}{1}

\textbf{Chatbot con Inteligencia Artificial}

En la parte inferior derecha en todas las ventanas de la aplicación se encuentra un botón que si hacemos click sobre él, nos abre una ventana de chat que nos permite hablar con una inteligencia artificial para hacer algunas preguntas sobre la liga.

\imagen{botonIA.png}{Botón que abre la ventana de texto de la IA}{0.2}

\imagen{vistaChatFinal.png}{Interfaz del chatbot inteligente}{0.4}