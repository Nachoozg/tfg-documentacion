\capitulo{1}{Introducción}

Durante varios años, el seguimiento del rendimiento deportivo a nivel de deportes escolares se ha basado en la utilización de métodos manuales como por ejemplo tablas con la clasificación en tablones de anuncios o en forma de papel. Sin embargo, con el auge de las tecnologías y la fácil accesibilidad a dispositivos que permiten ver páginas web por parte de todo el mundo, surge una gran oportunidad para digitalizar y optimizar la gestión y organización de estas ligas y hacer que todo esto sea más fácil de ver y gestionar.

En el mundo del tenis normalmente las competiciones que hay son torneos de corta duración en los que cada jugador va pasando por diferentes fases y eliminatorias hasta llegar a la final y tener opción a ganarla, pero en ese tipo de torneos se suele competir representando a la misma persona, es decir, cada jugador compite por él mismo y no por un equipo.

En este caso el planteamiento es algo diferente ya que la forma de competir será diferente a la clásica que se lleva a cabo en el tenis más profesional. Como es una liga entre colegios, su duración vendrá dada por el calendario escolar y no será en forma de torneo con una duración más corta. 

En esta liga se podrán inscribir los colegios que lo soliciten, pudiendo después cada uno de ellos inscribir a los jugadores que quieran. Los colegios competirán durante toda la temporada por ver cual es el que más puntos logra conseguir, asignando para cada partido el jugador que cada colegio elija.

Esta forma de plantear la competición permite que al jugar muchos partidos, jueguen muchos jugadores diferentes entre ellos y que aunque se enfrenten en un partido los mismos colegios que se enfrentaron en otro partido anterior, el resultado pueda ser muy diferente al poder ser otros jugadores los que se enfrenten en representación de ese colegio.

En una liga de tenis como esta, tenemos una competición en la que se inscriben colegios y juegan partidos entre ellos, aportando la victoria de un partido 2 puntos para la clasificación y la derrota 1 punto, al ser tenis, no existe el empate y por tanto se puntuará de esta manera.

Este proyecto se centra en el desarrollo de una aplicación web para la gestión de una liga de tenis escolar en Burgos. A diferencia de las plataformas tradicionales que solo permiten mostrar resultados, con esta aplicación se busca ofrecer una experiencia completa que combina la información, análisis y funcionalidades nuevas tanto para jugadores, como entrenadores, organizadores y público en general.

\imagen{mi-web.png}{Imagen de la página de inicio de la web}{1.0}

Este proyecto propone una solución innovadora utilizando tecnologías más modernas como angular para el front-end y .net para el back-end junto con una base de datos que permite almacenar y consultar información clave sobre los partidos, jugadores, colegios y la clasificación en una tabla global por cada colegio. La aplicación también contempla la integración de herramientas para facilitar el registro de resultados visualización de calendarios y estadísticas y un sistema de predicción de resultados basados en los datos históricos.

Uno de los elementos más destacados de la web es su enfoque en la accesibilidad y el dinamismo, permitiendo que cualquier usuario pueda seguir el desarrollo de la Liga en tiempo real y conocer qué jugadores son los mejores y descubrir cómo pueden terminar los siguientes enfrentamientos.

\section{Estructura de la memoria}\label{estructura-de-la-memoria}

La estructura que sigue la memoria es la siguiente:

\begin{itemize}
\tightlist
\item
  \textbf{Introducción:} descripción breve del problema que se resuelve y contexto del proyecto junto con la estructura de la memoria y de los anexos.
\item
  \textbf{Objetivos del proyecto:} lista con los objetivos que se quieren conseguir alcanzar con la realización del proyecto. De desglosan en los objetivos generales, técnicos y personales.
\item
  \textbf{Conceptos teóricos:} explicación sencilla sobre los conceptos teóricos que son necesarios para comprender correctamente el proyecto realizado.
\item
  \textbf{Técnicas y herramientas:} listado que explica las técnicas y las herramientas que se han utilizado durante el desarrollo del proyecto.
\item
  \textbf{Aspectos relevantes del desarrollo:} se muestran los aspectos más importantes que han tenido lugar durante el tiempo de desarrollo del proyecto.
\item
  \textbf{Trabajos relacionados:} investigación de trabajos similares o con relación al proyecto desarrollado y comparación de funcionalidades con estos.
\item
  \textbf{Conclusiones y líneas de trabajo futuras:} conclusiones obtenidas tras realizar el proyecto y posibles mejoras que se puedan realizar en actualizaciones futuras.
\end{itemize}



\section{Estructura de los anexos}\label{estructura-anexos}

\begin{itemize}
\tightlist
\item
  \textbf{Plan de proyecto:} descripción de la planificación del proyecto, tiene tanto la planificación temporal como el estudio de la viabilidad económica y legal.
\item
  \textbf{Especificación de requisitos:} se especifican los requisitos en función de los objetivos que se establecieron al inicio del proyecto, además de los que han surgido durante el proyecto.
\item
  \textbf{Diseño:} explicación del diseño que se han seguido para realizar la aplicación que forma el proyecto.
\item
  \textbf{Manual del programador:} explicación de conceptos más específicos y relacionados con la parte de desarrollo y programación del proyecto.
\item
  \textbf{Manual de usuario:} manual que sirve para comprender como se utiliza la aplicación con todas y cada una de las funcionalidades que aporta.
\item
  \textbf{Sostenibilización curricular:} documentación sobre los temas relacionados con la sostenibilidad que se llevan a cabo en el proyecto.
\end{itemize}


\section{Materiales adjuntos}\label{materiales-adjuntos}

Los materiales que se entregan adjuntos con la memoria son: 

\begin{itemize}
\tightlist
\item
	Aplicación web para la liga de tenis de los colegios de Burgos.
    Se puede acceder desde este enlace: \href{http://13.61.14.244/}{Liga de tenis de colegios de Burgos}
\item	
	Vídeo de demostración del funcionamiento. \href{https://youtu.be/CiM1DY7O88A}{Vídeo de demostración de funcionamiento}
\end{itemize}

Además, los siguientes recursos están accesibles a través de internet:

\begin{itemize}
\tightlist
\item
  Página web de la liga... PONER ENLACE CUANDO SE SUBA

\item Repositorio del front-end: 
    \href{https://github.com/Nachoozg/tfg}{LigaTenis front en GitHub}.
    
\item Repositorio del back-end: 
    \href{https://github.com/Nachoozg/tfg-back}{LigaTenis back en GitHub}.

\end{itemize}
