\apendice{Especificación de Requisitos}

\section{Introducción}
En este apéndice se desarrollarán todos los objetivos y los requisitos que tiene que cumplir la aplicación según lo que se ha establecido al principio del proyecto y todo lo nuevo que ha ido surgiendo mientras se realizaba el desarrollo.

\section{Objetivos generales}

Los objetivos generales que se acordaron al inicio del proyecto fueron:
\begin{itemize}
\tightlist
\item
  Desarrollar una aplicación web utilizando Angular como front-end, .NET como back-end.
\item
  Desarrollar una API para poder hacer la gestión de todos los datos y realizar operaciones sobre ellos.
\item
  Crear una base de datos MySQL en HeidiSQL para guardar los datos necesarios en tablas.
\item
  Implementar la autenticación de usuarios y restricciones por roles de cada usuario.
\item
  Integrar APIs para tener mapas interactivos en la web.
\item
  Integrar una forma de predecir los resultados de los próximos partidos.
\end{itemize}


\section{Catalogo de requisitos}
En este apartado se enumeran todos los requisitos funcionales y no funcionales de todo el proyecto.

\subsection{Requisitos funcionales}

\begin{itemize}
\tightlist
\item
  \textbf{RF-1 Login de usuarios:} la aplicación tiene que ser capaz de permitir a los colegios y a los árbitros autenticarse.
\item
  \textbf{RF-2 Registro de usuarios:} la aplicación tiene que ser capaz de permitir a los colegios y a los árbitros registrarse y poder almacenar sus datos en la base de datos.
\item
  \textbf{RF-3 Cierre de sesión de usuarios:} la aplicación debe permitir al usuario que tenga la sesión iniciada con sus credenciales en ese momento cerrar esa sesión.
\item
  \textbf{RF-4 Gestión de rol:} la aplicación debe tener restricciones de acceso a la modificación de datos en función del rol con el que se tenga iniciada la sesión.
\item
  \textbf{RF-5 Navegación de la aplicación:} la aplicación debe tener un menú que permita al usuario navegar entre las diferentes ventanas que la componen.
\item
  \textbf{RF-6 Página de inicio de la aplicación:} la aplicación debe tener una página principal que muestre una buena descripción sobre la información que se puede encontrar y sea atractiva.
\item
  \textbf{RF-7 Predicción de resultados:} la aplicación debe tener un algoritmo que permita predecir los resultados de los partidos que están por jugarse.
\item
  \textbf{RF-8 Visualización de clasificación:} la aplicación debe contener una sección en la que se pueda ver la información sobre la clasificación general de cada equipo en la liga.
\item
  \textbf{RF-9 Gestión de Colegios:} la aplicación debe ser capaz de gestionar los colegios inscritos en la liga.
  \begin{itemize}
  \tightlist
  \item
    \textbf{RF-9.1 Ver colegio:} el usuario debe poder ver la información detallada sobre el colegio.
  \item
    \textbf{RF-9.2 Añadir colegio:} el usuario que tenga el rol necesario debe ser capaz de agregar nuevos colegios con su nombre y detalles específicos.
  \item
    \textbf{RF-9.3 Editar colegio:} el usuario que tenga el rol necesario debe ser capaz de editar la información guardada de los colegios.
  \item
    \textbf{RF-9.4 Eliminar colegio:} el usuario que tenga el rol necesario debe ser capaz de eliminar la información guardada de los colegios.
  \end{itemize}
\item
  \textbf{RF-10 Gestión de Jugadores:} la aplicación debe ser capaz de gestionar los jugadores inscritos en la liga.
  \begin{itemize}
  \tightlist
  \item
    \textbf{RF-10.1 Ver jugador:} el usuario debe poder ver la información detallada sobre el jugador.
  \item
    \textbf{RF-10.2 Añadir jugador:} el usuario que tenga el rol necesario debe ser capaz de agregar nuevos jugadores con su nombre y detalles específicos.
  \item
    \textbf{RF-10.3 Editar jugador:} el usuario que tenga el rol necesario debe ser capaz de editar la información guardada de los jugadores.
  \item
    \textbf{RF-10.4 Eliminar jugador:} el usuario que tenga el rol necesario debe ser capaz de eliminar la información guardada de los jugadores.
  \end{itemize}
\item
  \textbf{RF-11 Gestión de Partidos:} la aplicación debe ser capaz de gestionar los partidos que se juegan en la liga.
  \begin{itemize}
  \tightlist
  \item
    \textbf{RF-11.1 Ver partido:} el usuario debe poder ver la información detallada sobre el partido.
  \item
    \textbf{RF-11.2 Añadir partido:} el usuario que tenga el rol necesario debe ser capaz de agregar nuevos partidos con sus equipos enfrentados y detalles específicos.
  \item
    \textbf{RF-11.3 Editar partido:} el usuario que tenga el rol necesario debe ser capaz de editar la información guardada de los partidos.
  \item
    \textbf{RF-11.4 Eliminar partido:} el usuario que tenga el rol necesario debe ser capaz de eliminar la información guardada de los partidos.
  \end{itemize}
\end{itemize}


\subsection{Requisitos no funcionales}

\begin{itemize}
\tightlist
\item
  \textbf{RNF-1 Usabilidad:} la aplicación debe tener una interfaz que sea visual e intuitiva, fácil de utilizar y con indicaciones de manejo de datos.
\item
  \textbf{RNF-2 Rendimiento:} la aplicación debe funcionar bajo tiempos viables, donde no haya largos tiempos de carga para ningún dispositivo.
\item
  \textbf{RNF-3 Responsividad:} la aplicación debe tener la capacidad de adaptarse a tamaños de pantalla diferentes.
\item
  \textbf{RNF-4 Escalabilidad:} la aplicación debe estar desarrollada de forma que puedan agregarse nuevas funcionalidades sin alterar las cargas de trabajo.
\item
  \textbf{RNF-5 Mantenibilidad:} la aplicación debe permitir realizar mantenimientos de manera simple y afectando al mínimo de componentes extra posible.
\item
  \textbf{RNF-6 Seguridad}: la aplicación debe proteger de la manera adecuada todos los datos que controla y sean sensibles como tokens de APIs o claves y credenciales de acceso además de proteger los datos de los usuarios.
\item
  \textbf{RNF-7 Disponibilidad}: la aplicación debe estar disponible para los usuarios todo el tiempo que sea posible, evitando tiempos de espera.
\end{itemize}

\newpage

\section{Especificación de requisitos}

En esta sección se desarrolla cada caso de uso y se muestra el diagrama de casos de uso final.

\subsection{Casos de uso}

\begin{longtable}[H]{@{}ll@{}}
\toprule
\begin{minipage}[b]{0.23\columnwidth}\raggedright\strut
\textbf{CU-01}\strut
\end{minipage} & \begin{minipage}[b]{0.71\columnwidth}\raggedright\strut
\textbf{Login de usuarios}\strut
\end{minipage}\tabularnewline
\midrule
\endhead
\begin{minipage}[t]{0.23\columnwidth}\raggedright\strut
\textbf{Versión}\strut
\end{minipage} & \begin{minipage}[t]{0.71\columnwidth}\raggedright\strut
1.0\strut
\end{minipage}\tabularnewline
\begin{minipage}[t]{0.23\columnwidth}\raggedright\strut
\textbf{Autor}\strut
\end{minipage} & \begin{minipage}[t]{0.71\columnwidth}\raggedright\strut
Ignacio Zaldo González\strut
\end{minipage}\tabularnewline
\begin{minipage}[t]{0.23\columnwidth}\raggedright\strut
\textbf{Requisitos asociados}\strut
\end{minipage} & \begin{minipage}[t]{0.71\columnwidth}\raggedright\strut
RF-1\strut
\end{minipage}\tabularnewline
\begin{minipage}[t]{0.23\columnwidth}\raggedright\strut
\textbf{Descripción}\strut
\end{minipage} & \begin{minipage}[t]{0.71\columnwidth}\raggedright\strut
Permite a un usuario (colegio o arbitro) iniciar sesión para poder modificar información.\strut
\end{minipage}\tabularnewline
\begin{minipage}[t]{0.23\columnwidth}\raggedright\strut
\textbf{Precondición}\strut
\end{minipage} & \begin{minipage}[t]{0.71\columnwidth}\raggedright\strut
Tener credenciales validadas que permitan acceder.\strut
\end{minipage}\tabularnewline
\begin{minipage}[t]{0.23\columnwidth}\raggedright\strut
\textbf{Acciones}\strut
\end{minipage} & \begin{minipage}[t]{0.71\columnwidth}\raggedright\strut
\begin{enumerate}
\def\labelenumi{\arabic{enumi}.}
\tightlist
\item
  El usuario entra en la aplicación.
\item
  El usuario pincha en el botón de inicio de sesión.
\item
  El usuario rellena el formulario con sus credenciales correctas.
\item
  El usuario pulsa el botón de inicio de sesión.
\item
  El sistema valida las credenciales introducidas.
\item
  Se redirige al usuario a la página principal.
\end{enumerate}\strut
\end{minipage}\tabularnewline
\begin{minipage}[t]{0.23\columnwidth}\raggedright\strut
\textbf{Postcondición}\strut
\end{minipage} & \begin{minipage}[t]{0.71\columnwidth}\raggedright\strut
El usuario ha accedido a su espacio.\strut
\end{minipage}\tabularnewline
\begin{minipage}[t]{0.23\columnwidth}\raggedright\strut
\textbf{Excepciones}\strut
\end{minipage} & \begin{minipage}[t]{0.71\columnwidth}\raggedright\strut
\begin{itemize}
\tightlist
\item
  Usuario inexistente en base de datos.
\item
  Credenciales introducidas no válidas.
\item
  Cuenta no activada.
\end{itemize}\strut
\end{minipage}\tabularnewline
\begin{minipage}[t]{0.23\columnwidth}\raggedright\strut
\textbf{Importancia}\strut
\end{minipage} & \begin{minipage}[t]{0.71\columnwidth}\raggedright\strut
Alta\strut
\end{minipage}\tabularnewline
\bottomrule
\caption{CU-01 Login de usuarios.}
\end{longtable}


\begin{longtable}[H]{@{}ll@{}}
\toprule
\begin{minipage}[b]{0.23\columnwidth}\raggedright\strut
\textbf{CU-02}\strut
\end{minipage} & \begin{minipage}[b]{0.71\columnwidth}\raggedright\strut
\textbf{Registro de usuarios}\strut
\end{minipage}\tabularnewline
\midrule
\endhead
\begin{minipage}[t]{0.23\columnwidth}\raggedright\strut
\textbf{Versión}\strut
\end{minipage} & \begin{minipage}[t]{0.71\columnwidth}\raggedright\strut
1.0\strut
\end{minipage}\tabularnewline
\begin{minipage}[t]{0.23\columnwidth}\raggedright\strut
\textbf{Autor}\strut
\end{minipage} & \begin{minipage}[t]{0.71\columnwidth}\raggedright\strut
Ignacio Zaldo González\strut
\end{minipage}\tabularnewline
\begin{minipage}[t]{0.23\columnwidth}\raggedright\strut
\textbf{Requisitos asociados}\strut
\end{minipage} & \begin{minipage}[t]{0.71\columnwidth}\raggedright\strut
RF-2\strut
\end{minipage}\tabularnewline
\begin{minipage}[t]{0.23\columnwidth}\raggedright\strut
\textbf{Descripción}\strut
\end{minipage} & \begin{minipage}[t]{0.71\columnwidth}\raggedright\strut
Permite a un usuario darse de alta introduciendo nombre, apellidos, contraseña y rol.\strut
\end{minipage}\tabularnewline
\begin{minipage}[t]{0.23\columnwidth}\raggedright\strut
\textbf{Precondición}\strut
\end{minipage} & \begin{minipage}[t]{0.71\columnwidth}\raggedright\strut
El usuario no debe existir aún en la base de datos.\strut
\end{minipage}\tabularnewline
\begin{minipage}[t]{0.23\columnwidth}\raggedright\strut
\textbf{Acciones}\strut
\end{minipage} & \begin{minipage}[t]{0.71\columnwidth}\raggedright\strut
\begin{enumerate}
\def\labelenumi{\arabic{enumi}.}
\tightlist
\item
  El usuario entra en la aplicación.
\item
  El usuario pincha en el botón de inicio de sesión.
\item
  El usuario pincha en la opción de registrarse.
\item
  El usuario rellena el formulario introduciendo su nombre, apellidos, contraseña y rol.
\item
  El sistema valida que el correo no exista y que todos los campos sean correctos.
\item
  Se guarda el nuevo usuario.
\item
  Se solicita validación de cuenta.
\item
  Se redirige al usuario a la página principal.
\end{enumerate}\strut
\end{minipage}\tabularnewline
\begin{minipage}[t]{0.23\columnwidth}\raggedright\strut
\textbf{Postcondición}\strut
\end{minipage} & \begin{minipage}[t]{0.71\columnwidth}\raggedright\strut
El nuevo usuario con las credenciales introducidas se ha guardado correctamente.\strut
\end{minipage}\tabularnewline
\begin{minipage}[t]{0.23\columnwidth}\raggedright\strut
\textbf{Excepciones}\strut
\end{minipage} & \begin{minipage}[t]{0.71\columnwidth}\raggedright\strut
\begin{itemize}
\tightlist
\item
  Usuario existente en base de datos.
\item
  Las contraseñas no coinciden.
\item
  Las contraseñas no tienen la seguridad suficiente.
\item
  Otras credenciales fallan.
\end{itemize}\strut
\end{minipage}\tabularnewline
\begin{minipage}[t]{0.23\columnwidth}\raggedright\strut
\textbf{Importancia}\strut
\end{minipage} & \begin{minipage}[t]{0.71\columnwidth}\raggedright\strut
Alta\strut
\end{minipage}\tabularnewline
\bottomrule
\caption{CU-02 Registro de usuarios.}
\end{longtable}


\begin{longtable}[H]{@{}ll@{}}
\toprule
\begin{minipage}[b]{0.23\columnwidth}\raggedright\strut
\textbf{CU-03}\strut
\end{minipage} & \begin{minipage}[b]{0.71\columnwidth}\raggedright\strut
\textbf{Cierre de sesión de usuarios}\strut
\end{minipage}\tabularnewline
\midrule
\endhead
\begin{minipage}[t]{0.23\columnwidth}\raggedright\strut
\textbf{Versión}\strut
\end{minipage} & \begin{minipage}[t]{0.71\columnwidth}\raggedright\strut
1.0\strut
\end{minipage}\tabularnewline
\begin{minipage}[t]{0.23\columnwidth}\raggedright\strut
\textbf{Autor}\strut
\end{minipage} & \begin{minipage}[t]{0.71\columnwidth}\raggedright\strut
Ignacio Zaldo González\strut
\end{minipage}\tabularnewline
\begin{minipage}[t]{0.23\columnwidth}\raggedright\strut
\textbf{Requisitos asociados}\strut
\end{minipage} & \begin{minipage}[t]{0.71\columnwidth}\raggedright\strut
RF-3\strut
\end{minipage}\tabularnewline
\begin{minipage}[t]{0.23\columnwidth}\raggedright\strut
\textbf{Descripción}\strut
\end{minipage} & \begin{minipage}[t]{0.71\columnwidth}\raggedright\strut
Permite a un usuario autenticado finalizar su sesión actual.\strut
\end{minipage}\tabularnewline
\begin{minipage}[t]{0.23\columnwidth}\raggedright\strut
\textbf{Precondición}\strut
\end{minipage} & \begin{minipage}[t]{0.71\columnwidth}\raggedright\strut
El usuario debe tener una sesión activa.\strut
\end{minipage}\tabularnewline
\begin{minipage}[t]{0.23\columnwidth}\raggedright\strut
\textbf{Acciones}\strut
\end{minipage} & \begin{minipage}[t]{0.71\columnwidth}\raggedright\strut
\begin{enumerate}
\def\labelenumi{\arabic{enumi}.}
\tightlist
\item
  El usuario pincha en el botón de cerrar sesión.
\item
  El sistema lanza una modal de aviso pidiendo la confirmación de cierre de sesión.
\item
  El usuario pincha en el botón de confirmar cierre de sesión.
\item
  El sistema redirige al usuario a la página principal.
\end{enumerate}\strut
\end{minipage}\tabularnewline
\begin{minipage}[t]{0.23\columnwidth}\raggedright\strut
\textbf{Postcondición}\strut
\end{minipage} & \begin{minipage}[t]{0.71\columnwidth}\raggedright\strut
La sesión del usuario tiene que quedar completamente cerrada.\strut
\end{minipage}\tabularnewline
\begin{minipage}[t]{0.23\columnwidth}\raggedright\strut
\textbf{Excepciones}\strut
\end{minipage} & \begin{minipage}[t]{0.71\columnwidth}\raggedright\strut
\begin{itemize}
\tightlist
\item
  Sesión no encontrada.
\end{itemize}\strut
\end{minipage}\tabularnewline
\begin{minipage}[t]{0.23\columnwidth}\raggedright\strut
\textbf{Importancia}\strut
\end{minipage} & \begin{minipage}[t]{0.71\columnwidth}\raggedright\strut
Media\strut
\end{minipage}\tabularnewline
\bottomrule
\caption{CU-03 Cierre de sesión de usuarios.}
\end{longtable}

\newpage

\begin{longtable}[H]{@{}ll@{}}
\toprule
\begin{minipage}[b]{0.23\columnwidth}\raggedright\strut
\textbf{CU-04}\strut
\end{minipage} & \begin{minipage}[b]{0.71\columnwidth}\raggedright\strut
\textbf{Gestión de rol}\strut
\end{minipage}\tabularnewline
\midrule
\endhead
\begin{minipage}[t]{0.23\columnwidth}\raggedright\strut
\textbf{Versión}\strut
\end{minipage} & \begin{minipage}[t]{0.71\columnwidth}\raggedright\strut
1.0\strut
\end{minipage}\tabularnewline
\begin{minipage}[t]{0.23\columnwidth}\raggedright\strut
\textbf{Autor}\strut
\end{minipage} & \begin{minipage}[t]{0.71\columnwidth}\raggedright\strut
Ignacio Zaldo González\strut
\end{minipage}\tabularnewline
\begin{minipage}[t]{0.23\columnwidth}\raggedright\strut
\textbf{Requisitos asociados}\strut
\end{minipage} & \begin{minipage}[t]{0.71\columnwidth}\raggedright\strut
RF-4\strut
\end{minipage}\tabularnewline
\begin{minipage}[t]{0.23\columnwidth}\raggedright\strut
\textbf{Descripción}\strut
\end{minipage} & \begin{minipage}[t]{0.71\columnwidth}\raggedright\strut
Permite o bloquea a un usuario la realización de determinadas acciones en función del rol que tenga.\strut
\end{minipage}\tabularnewline
\begin{minipage}[t]{0.23\columnwidth}\raggedright\strut
\textbf{Precondición}\strut
\end{minipage} & \begin{minipage}[t]{0.71\columnwidth}\raggedright\strut
El usuario debe estar registrado y con la sesión iniciada con un rol válido.\strut
\end{minipage}\tabularnewline
\begin{minipage}[t]{0.23\columnwidth}\raggedright\strut
\textbf{Acciones}\strut
\end{minipage} & \begin{minipage}[t]{0.71\columnwidth}\raggedright\strut
\begin{enumerate}
\def\labelenumi{\arabic{enumi}.}
\tightlist
\item
  El usuario entra en la aplicación.
\item
  El usuario pincha en el botón de inicio de sesión.
\item
  El usuario navega por las vistas de la aplicación.
\item
  Si el usuario tiene rol administrador, puede modificar todo y tiene visible toda la información.
\item
  Si el usuario tiene rol árbitro, puede ver y modificar toda la información sobre los partidos.
\item
  Si el usuario tiene rol colegio, puede ver y modificar toda la información relacionada sobre los colegios y los jugadores.
\end{enumerate}\strut
\end{minipage}\tabularnewline
\begin{minipage}[t]{0.23\columnwidth}\raggedright\strut
\textbf{Postcondición}\strut
\end{minipage} & \begin{minipage}[t]{0.71\columnwidth}\raggedright\strut
El usuario solo ve las opciones y puede ejecutar las acciones permitidas por su rol.\strut
\end{minipage}\tabularnewline
\begin{minipage}[t]{0.23\columnwidth}\raggedright\strut
\textbf{Excepciones}\strut
\end{minipage} & \begin{minipage}[t]{0.71\columnwidth}\raggedright\strut
\begin{itemize}
\tightlist
\item
  Usuario inexistente.
\item
  Rol no validado.
\item
  Rol incorrecto.
\end{itemize}\strut
\end{minipage}\tabularnewline
\begin{minipage}[t]{0.23\columnwidth}\raggedright\strut
\textbf{Importancia}\strut
\end{minipage} & \begin{minipage}[t]{0.71\columnwidth}\raggedright\strut
Alta\strut
\end{minipage}\tabularnewline
\bottomrule
\caption{CU-04 Gestión de rol.}
\end{longtable}


\begin{longtable}[H]{@{}ll@{}}
\toprule
\begin{minipage}[b]{0.23\columnwidth}\raggedright\strut
\textbf{CU-05}\strut
\end{minipage} & \begin{minipage}[b]{0.71\columnwidth}\raggedright\strut
\textbf{Navegación de la aplicación}\strut
\end{minipage}\tabularnewline
\midrule
\endhead
\begin{minipage}[t]{0.23\columnwidth}\raggedright\strut
\textbf{Versión}\strut
\end{minipage} & \begin{minipage}[t]{0.71\columnwidth}\raggedright\strut
1.0\strut
\end{minipage}\tabularnewline
\begin{minipage}[t]{0.23\columnwidth}\raggedright\strut
\textbf{Autor}\strut
\end{minipage} & \begin{minipage}[t]{0.71\columnwidth}\raggedright\strut
Ignacio Zaldo González\strut
\end{minipage}\tabularnewline
\begin{minipage}[t]{0.23\columnwidth}\raggedright\strut
\textbf{Requisitos asociados}\strut
\end{minipage} & \begin{minipage}[t]{0.71\columnwidth}\raggedright\strut
RF-5\strut
\end{minipage}\tabularnewline
\begin{minipage}[t]{0.23\columnwidth}\raggedright\strut
\textbf{Descripción}\strut
\end{minipage} & \begin{minipage}[t]{0.71\columnwidth}\raggedright\strut
Permite al usuario navegar por las diferentes ventanas y componentes de la aplicación.\strut
\end{minipage}\tabularnewline
\begin{minipage}[t]{0.23\columnwidth}\raggedright\strut
\textbf{Precondición}\strut
\end{minipage} & \begin{minipage}[t]{0.71\columnwidth}\raggedright\strut
Se tiene que haber accedido a la página web.\strut
\end{minipage}\tabularnewline
\begin{minipage}[t]{0.23\columnwidth}\raggedright\strut
\textbf{Acciones}\strut
\end{minipage} & \begin{minipage}[t]{0.71\columnwidth}\raggedright\strut
\begin{enumerate}
\def\labelenumi{\arabic{enumi}.}
\tightlist
\item
  El usuario entra en la aplicación.
\item
  Si está navegando con un dispositivo con pantalla pequeña, pincha en el menú desplegable para ver las posibles ventanas accesibles.
\item
  Si está navegando en una ventana grande, se dirige a la barra superior de navegación donde se muestran todas las ventanas de la aplicación.
\item
  El usuario puede acceder a la página de inicio.
\item
  El usuario puede acceder a la página de la clasificación.
\item
  El usuario puede acceder a la página de los colegios.
\item
  El usuario puede acceder a la página de los jugadores.
\item
  El usuario puede acceder a la página de los partidos.
\item
  El usuario puede acceder a la página de inicio de sesión.
\end{enumerate}\strut
\end{minipage}\tabularnewline
\begin{minipage}[t]{0.23\columnwidth}\raggedright\strut
\textbf{Postcondición}\strut
\end{minipage} & \begin{minipage}[t]{0.71\columnwidth}\raggedright\strut
Se redirige al usuario a la ventana que haya seleccionado.\strut
\end{minipage}\tabularnewline
\begin{minipage}[t]{0.23\columnwidth}\raggedright\strut
\textbf{Excepciones}\strut
\end{minipage} & \begin{minipage}[t]{0.71\columnwidth}\raggedright\strut
\end{minipage}\tabularnewline
\begin{minipage}[t]{0.23\columnwidth}\raggedright\strut
\textbf{Importancia}\strut
\end{minipage} & \begin{minipage}[t]{0.71\columnwidth}\raggedright\strut
Alta\strut
\end{minipage}\tabularnewline
\bottomrule
\caption{CU-05 Navegación de la aplicación.}
\end{longtable}


\begin{longtable}[H]{@{}ll@{}}
\toprule
\begin{minipage}[b]{0.23\columnwidth}\raggedright\strut
\textbf{CU-06}\strut
\end{minipage} & \begin{minipage}[b]{0.71\columnwidth}\raggedright\strut
\textbf{Página de inicio de la aplicación}\strut
\end{minipage}\tabularnewline
\midrule
\endhead
\begin{minipage}[t]{0.23\columnwidth}\raggedright\strut
\textbf{Versión}\strut
\end{minipage} & \begin{minipage}[t]{0.71\columnwidth}\raggedright\strut
1.0\strut
\end{minipage}\tabularnewline
\begin{minipage}[t]{0.23\columnwidth}\raggedright\strut
\textbf{Autor}\strut
\end{minipage} & \begin{minipage}[t]{0.71\columnwidth}\raggedright\strut
Ignacio Zaldo González\strut
\end{minipage}\tabularnewline
\begin{minipage}[t]{0.23\columnwidth}\raggedright\strut
\textbf{Requisitos asociados}\strut
\end{minipage} & \begin{minipage}[t]{0.71\columnwidth}\raggedright\strut
RF-6\strut
\end{minipage}\tabularnewline
\begin{minipage}[t]{0.23\columnwidth}\raggedright\strut
\textbf{Descripción}\strut
\end{minipage} & \begin{minipage}[t]{0.71\columnwidth}\raggedright\strut
Permite al usuario ver el contenido y acceder a todas las diferentes ventanas de manera intuitiva.\strut
\end{minipage}\tabularnewline
\begin{minipage}[t]{0.23\columnwidth}\raggedright\strut
\textbf{Precondición}\strut
\end{minipage} & \begin{minipage}[t]{0.71\columnwidth}\raggedright\strut
El usuario debe haber abierto la página web.\strut
\end{minipage}\tabularnewline
\begin{minipage}[t]{0.23\columnwidth}\raggedright\strut
\textbf{Acciones}\strut
\end{minipage} & \begin{minipage}[t]{0.71\columnwidth}\raggedright\strut
\begin{enumerate}
\def\labelenumi{\arabic{enumi}.}
\tightlist
\item
  El usuario entra en la aplicación.
\item
  El usuario tiene accesibles todas las ventanas de la aplicación desde una vista general en la página de inicio y puede ver el calendario y la información de la web.
\end{enumerate}\strut
\end{minipage}\tabularnewline
\begin{minipage}[t]{0.23\columnwidth}\raggedright\strut
\textbf{Postcondición}\strut
\end{minipage} & \begin{minipage}[t]{0.71\columnwidth}\raggedright\strut
Se redirige al usuario a la ventana que haya seleccionado.\strut
\end{minipage}\tabularnewline
\begin{minipage}[t]{0.23\columnwidth}\raggedright\strut
\textbf{Excepciones}\strut
\end{minipage} & \begin{minipage}[t]{0.71\columnwidth}\raggedright\strut
\end{minipage}\tabularnewline
\begin{minipage}[t]{0.23\columnwidth}\raggedright\strut
\textbf{Importancia}\strut
\end{minipage} & \begin{minipage}[t]{0.71\columnwidth}\raggedright\strut
Media\strut
\end{minipage}\tabularnewline
\bottomrule
\caption{CU-06 Página de inicio de la aplicación.}
\end{longtable}

\newpage

\begin{longtable}[H]{@{}ll@{}}
\toprule
\begin{minipage}[b]{0.23\columnwidth}\raggedright\strut
\textbf{CU-07}\strut
\end{minipage} & \begin{minipage}[b]{0.71\columnwidth}\raggedright\strut
\textbf{Predicción de resultados}\strut
\end{minipage}\tabularnewline
\midrule
\endhead
\begin{minipage}[t]{0.23\columnwidth}\raggedright\strut
\textbf{Versión}\strut
\end{minipage} & \begin{minipage}[t]{0.71\columnwidth}\raggedright\strut
1.0\strut
\end{minipage}\tabularnewline
\begin{minipage}[t]{0.23\columnwidth}\raggedright\strut
\textbf{Autor}\strut
\end{minipage} & \begin{minipage}[t]{0.71\columnwidth}\raggedright\strut
Ignacio Zaldo González\strut
\end{minipage}\tabularnewline
\begin{minipage}[t]{0.23\columnwidth}\raggedright\strut
\textbf{Requisitos asociados}\strut
\end{minipage} & \begin{minipage}[t]{0.71\columnwidth}\raggedright\strut
RF-7\strut
\end{minipage}\tabularnewline
\begin{minipage}[t]{0.23\columnwidth}\raggedright\strut
\textbf{Descripción}\strut
\end{minipage} & \begin{minipage}[t]{0.71\columnwidth}\raggedright\strut
Permite al usuario acceder a un chatbot inteligente que predice resultados.\strut
\end{minipage}\tabularnewline
\begin{minipage}[t]{0.23\columnwidth}\raggedright\strut
\textbf{Precondición}\strut
\end{minipage} & \begin{minipage}[t]{0.71\columnwidth}\raggedright\strut
El usuario debe haber accedido al chat de la web.\strut
\end{minipage}\tabularnewline
\begin{minipage}[t]{0.23\columnwidth}\raggedright\strut
\textbf{Acciones}\strut
\end{minipage} & \begin{minipage}[t]{0.71\columnwidth}\raggedright\strut
\begin{enumerate}
\def\labelenumi{\arabic{enumi}.}
\tightlist
\item
  El usuario entra en la aplicación.
\item
  El usuario pincha en el botón que activa el chatbot inteligente.
\item
  El usuario hace preguntas sobre el contenido de la liga de tenis.
\item
  El usuario recibe respuestas.
\item
  El usuario pregunta que partidos se jugarán próximamente.
\item
  El sistema envía su respuesta.
\item
  El usuario pregunta sobre cual es la predicción para uno de los partidos próximos.
\item
  El sistema responde con los datos desglosados de la predicción y devuelve porcentajes de victoria.
\end{enumerate}\strut
\end{minipage}\tabularnewline
\begin{minipage}[t]{0.23\columnwidth}\raggedright\strut
\textbf{Postcondición}\strut
\end{minipage} & \begin{minipage}[t]{0.71\columnwidth}\raggedright\strut
El usuario recibe respuesta a sus preguntas y obtiene la predicción.\strut
\end{minipage}\tabularnewline
\begin{minipage}[t]{0.23\columnwidth}\raggedright\strut
\textbf{Excepciones}\strut
\end{minipage} & \begin{minipage}[t]{0.71\columnwidth}\raggedright\strut
\begin{itemize}
\tightlist
\item
  Preguntas mal formadas.
\item
  Tipos de pregunta no válidas.
\end{itemize}\strut
\end{minipage}\tabularnewline
\begin{minipage}[t]{0.23\columnwidth}\raggedright\strut
\textbf{Importancia}\strut
\end{minipage} & \begin{minipage}[t]{0.71\columnwidth}\raggedright\strut
Baja\strut
\end{minipage}\tabularnewline
\bottomrule
\caption{CU-07 Predicción de resultados.}
\end{longtable}


\begin{longtable}[H]{@{}ll@{}}
\toprule
\begin{minipage}[b]{0.23\columnwidth}\raggedright\strut
\textbf{CU-08}\strut
\end{minipage} & \begin{minipage}[b]{0.71\columnwidth}\raggedright\strut
\textbf{Visualización de la clasificación}\strut
\end{minipage}\tabularnewline
\midrule
\endhead
\begin{minipage}[t]{0.23\columnwidth}\raggedright\strut
\textbf{Versión}\strut
\end{minipage} & \begin{minipage}[t]{0.71\columnwidth}\raggedright\strut
1.0\strut
\end{minipage}\tabularnewline
\begin{minipage}[t]{0.23\columnwidth}\raggedright\strut
\textbf{Autor}\strut
\end{minipage} & \begin{minipage}[t]{0.71\columnwidth}\raggedright\strut
Ignacio Zaldo González\strut
\end{minipage}\tabularnewline
\begin{minipage}[t]{0.23\columnwidth}\raggedright\strut
\textbf{Requisitos asociados}\strut
\end{minipage} & \begin{minipage}[t]{0.71\columnwidth}\raggedright\strut
RF-8\strut
\end{minipage}\tabularnewline
\begin{minipage}[t]{0.23\columnwidth}\raggedright\strut
\textbf{Descripción}\strut
\end{minipage} & \begin{minipage}[t]{0.71\columnwidth}\raggedright\strut
Permite al usuario ver la tabla de clasificación e interactuar con ella.\strut
\end{minipage}\tabularnewline
\begin{minipage}[t]{0.23\columnwidth}\raggedright\strut
\textbf{Precondición}\strut
\end{minipage} & \begin{minipage}[t]{0.71\columnwidth}\raggedright\strut
Haber accedido a la ventana de clasificación.\strut
\end{minipage}\tabularnewline
\begin{minipage}[t]{0.23\columnwidth}\raggedright\strut
\textbf{Acciones}\strut
\end{minipage} & \begin{minipage}[t]{0.71\columnwidth}\raggedright\strut
\begin{enumerate}
\def\labelenumi{\arabic{enumi}.}
\tightlist
\item
  El usuario entra en la aplicación.
\item
  El usuario pincha en el botón de clasificación de la barra de navegación.
\item
  El sistema redirige al usuario a la tabla de clasificación.
\item
  El usuario puede ver la tabla e interactuar con ella para visualizar todo el contenido que ofrece.
\end{enumerate}\strut
\end{minipage}\tabularnewline
\begin{minipage}[t]{0.23\columnwidth}\raggedright\strut
\textbf{Postcondición}\strut
\end{minipage} & \begin{minipage}[t]{0.71\columnwidth}\raggedright\strut
El usuario ve la ventana de clasificación.\strut
\end{minipage}\tabularnewline
\begin{minipage}[t]{0.23\columnwidth}\raggedright\strut
\textbf{Excepciones}\strut
\end{minipage} & \begin{minipage}[t]{0.71\columnwidth}\raggedright\strut
\begin{itemize}
\tightlist
\item
  Se pincha en otro botón y redirige a otra ventana.
\end{itemize}\strut
\end{minipage}\tabularnewline
\begin{minipage}[t]{0.23\columnwidth}\raggedright\strut
\textbf{Importancia}\strut
\end{minipage} & \begin{minipage}[t]{0.71\columnwidth}\raggedright\strut
Media\strut
\end{minipage}\tabularnewline
\bottomrule
\caption{CU-08 Visualización de la clasificación.}
\end{longtable}

\newpage

\begin{longtable}[H]{@{}ll@{}}
\toprule
\begin{minipage}[b]{0.23\columnwidth}\raggedright\strut
\textbf{CU-09}\strut
\end{minipage} & \begin{minipage}[b]{0.71\columnwidth}\raggedright\strut
\textbf{Gestión de Colegios}\strut
\end{minipage}\tabularnewline
\midrule
\endhead
\begin{minipage}[t]{0.23\columnwidth}\raggedright\strut
\textbf{Versión}\strut
\end{minipage} & \begin{minipage}[t]{0.71\columnwidth}\raggedright\strut
1.0\strut
\end{minipage}\tabularnewline
\begin{minipage}[t]{0.23\columnwidth}\raggedright\strut
\textbf{Autor}\strut
\end{minipage} & \begin{minipage}[t]{0.71\columnwidth}\raggedright\strut
Ignacio Zaldo González\strut
\end{minipage}\tabularnewline
\begin{minipage}[t]{0.23\columnwidth}\raggedright\strut
\textbf{Requisitos asociados}\strut
\end{minipage} & \begin{minipage}[t]{0.71\columnwidth}\raggedright\strut
RF-9, RF-9.1, RF-9.2, RF-9.3, RF-9.4\strut
\end{minipage}\tabularnewline
\begin{minipage}[t]{0.23\columnwidth}\raggedright\strut
\textbf{Descripción}\strut
\end{minipage} & \begin{minipage}[t]{0.71\columnwidth}\raggedright\strut
Permite al usuario con el rol necesario gestionar la información de los colegios.\strut
\end{minipage}\tabularnewline
\begin{minipage}[t]{0.23\columnwidth}\raggedright\strut
\textbf{Precondición}\strut
\end{minipage} & \begin{minipage}[t]{0.71\columnwidth}\raggedright\strut
Debe haber colegios inscritos.\strut
\end{minipage}\tabularnewline
\begin{minipage}[t]{0.23\columnwidth}\raggedright\strut
\textbf{Acciones}\strut
\end{minipage} & \begin{minipage}[t]{0.71\columnwidth}\raggedright\strut
\begin{enumerate}
\def\labelenumi{\arabic{enumi}.}
\tightlist
\item
  El usuario entra en la aplicación.
\item
  El usuario se dirige a la ventana de colegios.
\item
  El sistema devuelve la lista de todos los colegios.
\item
  El usuario elige que acción realizar.
\end{enumerate}\strut
\end{minipage}\tabularnewline
\begin{minipage}[t]{0.23\columnwidth}\raggedright\strut
\textbf{Postcondición}\strut
\end{minipage} & \begin{minipage}[t]{0.71\columnwidth}\raggedright\strut
La tabla de clasificación se muestra correctamente.\strut
\end{minipage}\tabularnewline
\begin{minipage}[t]{0.23\columnwidth}\raggedright\strut
\textbf{Excepciones}\strut
\end{minipage} & \begin{minipage}[t]{0.71\columnwidth}\raggedright\strut
\begin{itemize}
\tightlist
\item
  Error al cargar la tabla.
\item
  Se ha pinchado en una ventana diferente.
\end{itemize}\strut
\end{minipage}\tabularnewline
\begin{minipage}[t]{0.23\columnwidth}\raggedright\strut
\textbf{Importancia}\strut
\end{minipage} & \begin{minipage}[t]{0.71\columnwidth}\raggedright\strut
Alta\strut
\end{minipage}\tabularnewline
\bottomrule
\caption{CU-09 Gestión de Colegios.}
\end{longtable}

\newpage

\begin{longtable}[H]{@{}ll@{}}
\toprule
\begin{minipage}[b]{0.23\columnwidth}\raggedright\strut
\textbf{CU-10}\strut
\end{minipage} & \begin{minipage}[b]{0.71\columnwidth}\raggedright\strut
\textbf{Ver colegio}\strut
\end{minipage}\tabularnewline
\midrule
\endhead
\begin{minipage}[t]{0.23\columnwidth}\raggedright\strut
\textbf{Versión}\strut
\end{minipage} & \begin{minipage}[t]{0.71\columnwidth}\raggedright\strut
1.0\strut
\end{minipage}\tabularnewline
\begin{minipage}[t]{0.23\columnwidth}\raggedright\strut
\textbf{Autor}\strut
\end{minipage} & \begin{minipage}[t]{0.71\columnwidth}\raggedright\strut
Ignacio Zaldo González\strut
\end{minipage}\tabularnewline
\begin{minipage}[t]{0.23\columnwidth}\raggedright\strut
\textbf{Requisitos asociados}\strut
\end{minipage} & \begin{minipage}[t]{0.71\columnwidth}\raggedright\strut
RF-9.1\strut
\end{minipage}\tabularnewline
\begin{minipage}[t]{0.23\columnwidth}\raggedright\strut
\textbf{Descripción}\strut
\end{minipage} & \begin{minipage}[t]{0.71\columnwidth}\raggedright\strut
Permite al usuario visualizar la información de cada colegio.\strut
\end{minipage}\tabularnewline
\begin{minipage}[t]{0.23\columnwidth}\raggedright\strut
\textbf{Precondición}\strut
\end{minipage} & \begin{minipage}[t]{0.71\columnwidth}\raggedright\strut
Debe haber colegios inscritos.\strut
\end{minipage}\tabularnewline
\begin{minipage}[t]{0.23\columnwidth}\raggedright\strut
\textbf{Acciones}\strut
\end{minipage} & \begin{minipage}[t]{0.71\columnwidth}\raggedright\strut
\begin{enumerate}
\def\labelenumi{\arabic{enumi}.}
\tightlist
\item
  El usuario se dirige a la ventana de colegios.
\item
  El usuario pincha en el nombre del colegio para expandir su información.
\item
  El sistema devuelve la tarjeta de información del colegio.
\item
  El usuario visualiza la información.
\end{enumerate}\strut
\end{minipage}\tabularnewline
\begin{minipage}[t]{0.23\columnwidth}\raggedright\strut
\textbf{Postcondición}\strut
\end{minipage} & \begin{minipage}[t]{0.71\columnwidth}\raggedright\strut
La información detallada del colegio se muestra correctamente.\strut
\end{minipage}\tabularnewline
\begin{minipage}[t]{0.23\columnwidth}\raggedright\strut
\textbf{Excepciones}\strut
\end{minipage} & \begin{minipage}[t]{0.71\columnwidth}\raggedright\strut
\begin{itemize}
\tightlist
\item
  Error al cargar la información.
\item
  No existe el colegio.
\end{itemize}\strut
\end{minipage}\tabularnewline
\begin{minipage}[t]{0.23\columnwidth}\raggedright\strut
\textbf{Importancia}\strut
\end{minipage} & \begin{minipage}[t]{0.71\columnwidth}\raggedright\strut
Alta\strut
\end{minipage}\tabularnewline
\bottomrule
\caption{CU-10 Ver colegio.}
\end{longtable}

\newpage

\begin{longtable}[H]{@{}ll@{}}
\toprule
\begin{minipage}[b]{0.23\columnwidth}\raggedright\strut
\textbf{CU-11}\strut
\end{minipage} & \begin{minipage}[b]{0.71\columnwidth}\raggedright\strut
\textbf{Añadir colegio}\strut
\end{minipage}\tabularnewline
\midrule
\endhead
\begin{minipage}[t]{0.23\columnwidth}\raggedright\strut
\textbf{Versión}\strut
\end{minipage} & \begin{minipage}[t]{0.71\columnwidth}\raggedright\strut
1.0\strut
\end{minipage}\tabularnewline
\begin{minipage}[t]{0.23\columnwidth}\raggedright\strut
\textbf{Autor}\strut
\end{minipage} & \begin{minipage}[t]{0.71\columnwidth}\raggedright\strut
Ignacio Zaldo González\strut
\end{minipage}\tabularnewline
\begin{minipage}[t]{0.23\columnwidth}\raggedright\strut
\textbf{Requisitos asociados}\strut
\end{minipage} & \begin{minipage}[t]{0.71\columnwidth}\raggedright\strut
RF-9.2\strut
\end{minipage}\tabularnewline
\begin{minipage}[t]{0.23\columnwidth}\raggedright\strut
\textbf{Descripción}\strut
\end{minipage} & \begin{minipage}[t]{0.71\columnwidth}\raggedright\strut
Permite al usuario con los roles necesarios añadir un nuevo colegio.\strut
\end{minipage}\tabularnewline
\begin{minipage}[t]{0.23\columnwidth}\raggedright\strut
\textbf{Precondición}\strut
\end{minipage} & \begin{minipage}[t]{0.71\columnwidth}\raggedright\strut
La sesión debe estar iniciada con un rol válido.\strut
\end{minipage}\tabularnewline
\begin{minipage}[t]{0.23\columnwidth}\raggedright\strut
\textbf{Acciones}\strut
\end{minipage} & \begin{minipage}[t]{0.71\columnwidth}\raggedright\strut
\begin{enumerate}
\def\labelenumi{\arabic{enumi}.}
\tightlist
\item
  El usuario se dirige a la ventana de colegios.
\item
  El usuario pincha en el botón de añadir colegio.
\item
  El sistema muestra el formulario con los datos necesarios para añadir el nuevo colegio.
\item
  El usuario introduce la información.
\item
  El usuario pincha en el botón de guardar.
\item
  El sistema lanza la tarjeta de aviso de nuevo colegio agregado.
\item
  El usuario puede ver el nuevo colegio en la lista.
\end{enumerate}\strut
\end{minipage}\tabularnewline
\begin{minipage}[t]{0.23\columnwidth}\raggedright\strut
\textbf{Postcondición}\strut
\end{minipage} & \begin{minipage}[t]{0.71\columnwidth}\raggedright\strut
Debe aparecer el nuevo colegio añadido en la lista de colegios.\strut
\end{minipage}\tabularnewline
\begin{minipage}[t]{0.23\columnwidth}\raggedright\strut
\textbf{Excepciones}\strut
\end{minipage} & \begin{minipage}[t]{0.71\columnwidth}\raggedright\strut
\begin{itemize}
\tightlist
\item
  Error al guardar la información.
\item
  Datos introducidos incorrectos.
\end{itemize}\strut
\end{minipage}\tabularnewline
\begin{minipage}[t]{0.23\columnwidth}\raggedright\strut
\textbf{Importancia}\strut
\end{minipage} & \begin{minipage}[t]{0.71\columnwidth}\raggedright\strut
Alta\strut
\end{minipage}\tabularnewline
\bottomrule
\caption{CU-11 Añadir colegio.}
\end{longtable}

\newpage

\begin{longtable}[H]{@{}ll@{}}
\toprule
\begin{minipage}[b]{0.23\columnwidth}\raggedright\strut
\textbf{CU-12}\strut
\end{minipage} & \begin{minipage}[b]{0.71\columnwidth}\raggedright\strut
\textbf{Editar colegio}\strut
\end{minipage}\tabularnewline
\midrule
\endhead
\begin{minipage}[t]{0.23\columnwidth}\raggedright\strut
\textbf{Versión}\strut
\end{minipage} & \begin{minipage}[t]{0.71\columnwidth}\raggedright\strut
1.0\strut
\end{minipage}\tabularnewline
\begin{minipage}[t]{0.23\columnwidth}\raggedright\strut
\textbf{Autor}\strut
\end{minipage} & \begin{minipage}[t]{0.71\columnwidth}\raggedright\strut
Ignacio Zaldo González\strut
\end{minipage}\tabularnewline
\begin{minipage}[t]{0.23\columnwidth}\raggedright\strut
\textbf{Requisitos asociados}\strut
\end{minipage} & \begin{minipage}[t]{0.71\columnwidth}\raggedright\strut
RF-9.3\strut
\end{minipage}\tabularnewline
\begin{minipage}[t]{0.23\columnwidth}\raggedright\strut
\textbf{Descripción}\strut
\end{minipage} & \begin{minipage}[t]{0.71\columnwidth}\raggedright\strut
Permite al usuario con los roles necesarios editar la información de un colegio.\strut
\end{minipage}\tabularnewline
\begin{minipage}[t]{0.23\columnwidth}\raggedright\strut
\textbf{Precondición}\strut
\end{minipage} & \begin{minipage}[t]{0.71\columnwidth}\raggedright\strut
La sesión debe estar iniciada con un rol válido.\strut
\end{minipage}\tabularnewline
\begin{minipage}[t]{0.23\columnwidth}\raggedright\strut
\textbf{Acciones}\strut
\end{minipage} & \begin{minipage}[t]{0.71\columnwidth}\raggedright\strut
\begin{enumerate}
\def\labelenumi{\arabic{enumi}.}
\tightlist
\item
  El usuario se dirige a la ventana de colegios.
\item
  El usuario pincha en el botón de editar colegio.
\item
  El sistema muestra el formulario con los datos que se pueden editar para ese colegio seleccionado.
\item
  El usuario edita la información.
\item
  El usuario pincha en el botón de guardar.
\item
  El sistema lanza la tarjeta de aviso de datos del colegio modificados.
\item
  El usuario puede ver el colegio con los datos nuevos en la lista.
\end{enumerate}\strut
\end{minipage}\tabularnewline
\begin{minipage}[t]{0.23\columnwidth}\raggedright\strut
\textbf{Postcondición}\strut
\end{minipage} & \begin{minipage}[t]{0.71\columnwidth}\raggedright\strut
Debe aparecer el colegio con los datos modificados en la lista de colegios.\strut
\end{minipage}\tabularnewline
\begin{minipage}[t]{0.23\columnwidth}\raggedright\strut
\textbf{Excepciones}\strut
\end{minipage} & \begin{minipage}[t]{0.71\columnwidth}\raggedright\strut
\begin{itemize}
\tightlist
\item
  Error al modificar la información.
\item
  Datos modificados incorrectos.
\end{itemize}\strut
\end{minipage}\tabularnewline
\begin{minipage}[t]{0.23\columnwidth}\raggedright\strut
\textbf{Importancia}\strut
\end{minipage} & \begin{minipage}[t]{0.71\columnwidth}\raggedright\strut
Alta\strut
\end{minipage}\tabularnewline
\bottomrule
\caption{CU-12 Editar colegio.}
\end{longtable}

\newpage

\begin{longtable}[H]{@{}ll@{}}
\toprule
\begin{minipage}[b]{0.23\columnwidth}\raggedright\strut
\textbf{CU-13}\strut
\end{minipage} & \begin{minipage}[b]{0.71\columnwidth}\raggedright\strut
\textbf{Eliminar colegio}\strut
\end{minipage}\tabularnewline
\midrule
\endhead
\begin{minipage}[t]{0.23\columnwidth}\raggedright\strut
\textbf{Versión}\strut
\end{minipage} & \begin{minipage}[t]{0.71\columnwidth}\raggedright\strut
1.0\strut
\end{minipage}\tabularnewline
\begin{minipage}[t]{0.23\columnwidth}\raggedright\strut
\textbf{Autor}\strut
\end{minipage} & \begin{minipage}[t]{0.71\columnwidth}\raggedright\strut
Ignacio Zaldo González\strut
\end{minipage}\tabularnewline
\begin{minipage}[t]{0.23\columnwidth}\raggedright\strut
\textbf{Requisitos asociados}\strut
\end{minipage} & \begin{minipage}[t]{0.71\columnwidth}\raggedright\strut
RF-9.4\strut
\end{minipage}\tabularnewline
\begin{minipage}[t]{0.23\columnwidth}\raggedright\strut
\textbf{Descripción}\strut
\end{minipage} & \begin{minipage}[t]{0.71\columnwidth}\raggedright\strut
Permite al usuario con los roles necesarios eliminar un colegio.\strut
\end{minipage}\tabularnewline
\begin{minipage}[t]{0.23\columnwidth}\raggedright\strut
\textbf{Precondición}\strut
\end{minipage} & \begin{minipage}[t]{0.71\columnwidth}\raggedright\strut
La sesión debe estar iniciada con un rol válido.\strut
\end{minipage}\tabularnewline
\begin{minipage}[t]{0.23\columnwidth}\raggedright\strut
\textbf{Acciones}\strut
\end{minipage} & \begin{minipage}[t]{0.71\columnwidth}\raggedright\strut
\begin{enumerate}
\def\labelenumi{\arabic{enumi}.}
\tightlist
\item
  El usuario se dirige a la ventana de colegios.
\item
  El usuario pincha en el botón de eliminar colegio.
\item
  El sistema muestra la modal de aviso de eliminación permanente de los datos.
\item
  El usuario pincha en el botón de aceptar.
\item
  El sistema lanza la tarjeta de aviso de datos del colegio eliminados.
\item
  El usuario ya no puede ver la información de ese colegio en la lista.
\end{enumerate}\strut
\end{minipage}\tabularnewline
\begin{minipage}[t]{0.23\columnwidth}\raggedright\strut
\textbf{Postcondición}\strut
\end{minipage} & \begin{minipage}[t]{0.71\columnwidth}\raggedright\strut
Debe desaparecer el colegio de la lista de colegios.\strut
\end{minipage}\tabularnewline
\begin{minipage}[t]{0.23\columnwidth}\raggedright\strut
\textbf{Excepciones}\strut
\end{minipage} & \begin{minipage}[t]{0.71\columnwidth}\raggedright\strut
\begin{itemize}
\tightlist
\item
  Error al eliminar la información.
\end{itemize}\strut
\end{minipage}\tabularnewline
\begin{minipage}[t]{0.23\columnwidth}\raggedright\strut
\textbf{Importancia}\strut
\end{minipage} & \begin{minipage}[t]{0.71\columnwidth}\raggedright\strut
Alta\strut
\end{minipage}\tabularnewline
\bottomrule
\caption{CU-13 Eliminar colegio.}
\end{longtable}

\newpage

\begin{longtable}[H]{@{}ll@{}}
\toprule
\begin{minipage}[b]{0.23\columnwidth}\raggedright\strut
\textbf{CU-14}\strut
\end{minipage} & \begin{minipage}[b]{0.71\columnwidth}\raggedright\strut
\textbf{Gestión de Jugadores}\strut
\end{minipage}\tabularnewline
\midrule
\endhead
\begin{minipage}[t]{0.23\columnwidth}\raggedright\strut
\textbf{Versión}\strut
\end{minipage} & \begin{minipage}[t]{0.71\columnwidth}\raggedright\strut
1.0\strut
\end{minipage}\tabularnewline
\begin{minipage}[t]{0.23\columnwidth}\raggedright\strut
\textbf{Autor}\strut
\end{minipage} & \begin{minipage}[t]{0.71\columnwidth}\raggedright\strut
Ignacio Zaldo González\strut
\end{minipage}\tabularnewline
\begin{minipage}[t]{0.23\columnwidth}\raggedright\strut
\textbf{Requisitos asociados}\strut
\end{minipage} & \begin{minipage}[t]{0.71\columnwidth}\raggedright\strut
RF-10, RF-10.1, RF-10.2, RF-10.3, RF-10.4\strut
\end{minipage}\tabularnewline
\begin{minipage}[t]{0.23\columnwidth}\raggedright\strut
\textbf{Descripción}\strut
\end{minipage} & \begin{minipage}[t]{0.71\columnwidth}\raggedright\strut
Permite al usuario con el rol necesario gestionar la información de los jugadores.\strut
\end{minipage}\tabularnewline
\begin{minipage}[t]{0.23\columnwidth}\raggedright\strut
\textbf{Precondición}\strut
\end{minipage} & \begin{minipage}[t]{0.71\columnwidth}\raggedright\strut
Debe haber jugadores inscritos.\strut
\end{minipage}\tabularnewline
\begin{minipage}[t]{0.23\columnwidth}\raggedright\strut
\textbf{Acciones}\strut
\end{minipage} & \begin{minipage}[t]{0.71\columnwidth}\raggedright\strut
\begin{enumerate}
\def\labelenumi{\arabic{enumi}.}
\tightlist
\item
  El usuario entra en la aplicación.
\item
  El usuario se dirige a la ventana de jugadores.
\item
  El sistema devuelve la lista de todos los jugadores.
\item
  El usuario elige que acción realizar.
\end{enumerate}\strut
\end{minipage}\tabularnewline
\begin{minipage}[t]{0.23\columnwidth}\raggedright\strut
\textbf{Postcondición}\strut
\end{minipage} & \begin{minipage}[t]{0.71\columnwidth}\raggedright\strut
La lista de jugadores se muestra correctamente.\strut
\end{minipage}\tabularnewline
\begin{minipage}[t]{0.23\columnwidth}\raggedright\strut
\textbf{Excepciones}\strut
\end{minipage} & \begin{minipage}[t]{0.71\columnwidth}\raggedright\strut
\begin{itemize}
\tightlist
\item
  Error al cargar la lista.
\item
  Se ha pinchado en una ventana diferente.
\end{itemize}\strut
\end{minipage}\tabularnewline
\begin{minipage}[t]{0.23\columnwidth}\raggedright\strut
\textbf{Importancia}\strut
\end{minipage} & \begin{minipage}[t]{0.71\columnwidth}\raggedright\strut
Alta\strut
\end{minipage}\tabularnewline
\bottomrule
\caption{CU-14 Gestión de Jugadores.}
\end{longtable}

\newpage

\begin{longtable}[H]{@{}ll@{}}
\toprule
\begin{minipage}[b]{0.23\columnwidth}\raggedright\strut
\textbf{CU-15}\strut
\end{minipage} & \begin{minipage}[b]{0.71\columnwidth}\raggedright\strut
\textbf{Ver jugador}\strut
\end{minipage}\tabularnewline
\midrule
\endhead
\begin{minipage}[t]{0.23\columnwidth}\raggedright\strut
\textbf{Versión}\strut
\end{minipage} & \begin{minipage}[t]{0.71\columnwidth}\raggedright\strut
1.0\strut
\end{minipage}\tabularnewline
\begin{minipage}[t]{0.23\columnwidth}\raggedright\strut
\textbf{Autor}\strut
\end{minipage} & \begin{minipage}[t]{0.71\columnwidth}\raggedright\strut
Ignacio Zaldo González\strut
\end{minipage}\tabularnewline
\begin{minipage}[t]{0.23\columnwidth}\raggedright\strut
\textbf{Requisitos asociados}\strut
\end{minipage} & \begin{minipage}[t]{0.71\columnwidth}\raggedright\strut
RF-10.1\strut
\end{minipage}\tabularnewline
\begin{minipage}[t]{0.23\columnwidth}\raggedright\strut
\textbf{Descripción}\strut
\end{minipage} & \begin{minipage}[t]{0.71\columnwidth}\raggedright\strut
Permite al usuario visualizar la información de cada jugador.\strut
\end{minipage}\tabularnewline
\begin{minipage}[t]{0.23\columnwidth}\raggedright\strut
\textbf{Precondición}\strut
\end{minipage} & \begin{minipage}[t]{0.71\columnwidth}\raggedright\strut
Debe haber jugadores inscritos.\strut
\end{minipage}\tabularnewline
\begin{minipage}[t]{0.23\columnwidth}\raggedright\strut
\textbf{Acciones}\strut
\end{minipage} & \begin{minipage}[t]{0.71\columnwidth}\raggedright\strut
\begin{enumerate}
\def\labelenumi{\arabic{enumi}.}
\tightlist
\item
  El usuario se dirige a la ventana de jugadores.
\item
  El usuario pincha en el nombre del jugador para expandir su información.
\item
  El sistema devuelve la tarjeta de información del jugador.
\item
  El usuario visualiza la información.
\end{enumerate}\strut
\end{minipage}\tabularnewline
\begin{minipage}[t]{0.23\columnwidth}\raggedright\strut
\textbf{Postcondición}\strut
\end{minipage} & \begin{minipage}[t]{0.71\columnwidth}\raggedright\strut
La información detallada del jugador se muestra correctamente.\strut
\end{minipage}\tabularnewline
\begin{minipage}[t]{0.23\columnwidth}\raggedright\strut
\textbf{Excepciones}\strut
\end{minipage} & \begin{minipage}[t]{0.71\columnwidth}\raggedright\strut
\begin{itemize}
\tightlist
\item
  Error al cargar la información.
\item
  No existe el jugador.
\end{itemize}\strut
\end{minipage}\tabularnewline
\begin{minipage}[t]{0.23\columnwidth}\raggedright\strut
\textbf{Importancia}\strut
\end{minipage} & \begin{minipage}[t]{0.71\columnwidth}\raggedright\strut
Alta\strut
\end{minipage}\tabularnewline
\bottomrule
\caption{CU-15 Ver jugador.}
\end{longtable}

\newpage

\begin{longtable}[H]{@{}ll@{}}
\toprule
\begin{minipage}[b]{0.23\columnwidth}\raggedright\strut
\textbf{CU-16}\strut
\end{minipage} & \begin{minipage}[b]{0.71\columnwidth}\raggedright\strut
\textbf{Añadir jugador}\strut
\end{minipage}\tabularnewline
\midrule
\endhead
\begin{minipage}[t]{0.23\columnwidth}\raggedright\strut
\textbf{Versión}\strut
\end{minipage} & \begin{minipage}[t]{0.71\columnwidth}\raggedright\strut
1.0\strut
\end{minipage}\tabularnewline
\begin{minipage}[t]{0.23\columnwidth}\raggedright\strut
\textbf{Autor}\strut
\end{minipage} & \begin{minipage}[t]{0.71\columnwidth}\raggedright\strut
Ignacio Zaldo González\strut
\end{minipage}\tabularnewline
\begin{minipage}[t]{0.23\columnwidth}\raggedright\strut
\textbf{Requisitos asociados}\strut
\end{minipage} & \begin{minipage}[t]{0.71\columnwidth}\raggedright\strut
RF-10.2\strut
\end{minipage}\tabularnewline
\begin{minipage}[t]{0.23\columnwidth}\raggedright\strut
\textbf{Descripción}\strut
\end{minipage} & \begin{minipage}[t]{0.71\columnwidth}\raggedright\strut
Permite al usuario con los roles necesarios añadir un nuevo jugador.\strut
\end{minipage}\tabularnewline
\begin{minipage}[t]{0.23\columnwidth}\raggedright\strut
\textbf{Precondición}\strut
\end{minipage} & \begin{minipage}[t]{0.71\columnwidth}\raggedright\strut
La sesión debe estar iniciada con un rol válido.\strut
\end{minipage}\tabularnewline
\begin{minipage}[t]{0.23\columnwidth}\raggedright\strut
\textbf{Acciones}\strut
\end{minipage} & \begin{minipage}[t]{0.71\columnwidth}\raggedright\strut
\begin{enumerate}
\def\labelenumi{\arabic{enumi}.}
\tightlist
\item
  El usuario se dirige a la ventana de jugadores.
\item
  El usuario pincha en el botón de añadir jugador.
\item
  El sistema muestra el formulario con los datos necesarios para añadir el nuevo jugador.
\item
  El usuario introduce la información.
\item
  El usuario pincha en el botón de guardar.
\item
  El sistema lanza la tarjeta de aviso de nuevo jugador agregado.
\item
  El usuario puede ver el nuevo jugador en la lista.
\end{enumerate}\strut
\end{minipage}\tabularnewline
\begin{minipage}[t]{0.23\columnwidth}\raggedright\strut
\textbf{Postcondición}\strut
\end{minipage} & \begin{minipage}[t]{0.71\columnwidth}\raggedright\strut
Debe aparecer el nuevo jugador añadido en la lista de jugadores.\strut
\end{minipage}\tabularnewline
\begin{minipage}[t]{0.23\columnwidth}\raggedright\strut
\textbf{Excepciones}\strut
\end{minipage} & \begin{minipage}[t]{0.71\columnwidth}\raggedright\strut
\begin{itemize}
\tightlist
\item
  Error al guardar la información.
\item
  Datos introducidos incorrectos.
\end{itemize}\strut
\end{minipage}\tabularnewline
\begin{minipage}[t]{0.23\columnwidth}\raggedright\strut
\textbf{Importancia}\strut
\end{minipage} & \begin{minipage}[t]{0.71\columnwidth}\raggedright\strut
Alta\strut
\end{minipage}\tabularnewline
\bottomrule
\caption{CU-16 Añadir jugador.}
\end{longtable}

\newpage

\begin{longtable}[H]{@{}ll@{}}
\toprule
\begin{minipage}[b]{0.23\columnwidth}\raggedright\strut
\textbf{CU-17}\strut
\end{minipage} & \begin{minipage}[b]{0.71\columnwidth}\raggedright\strut
\textbf{Editar jugador}\strut
\end{minipage}\tabularnewline
\midrule
\endhead
\begin{minipage}[t]{0.23\columnwidth}\raggedright\strut
\textbf{Versión}\strut
\end{minipage} & \begin{minipage}[t]{0.71\columnwidth}\raggedright\strut
1.0\strut
\end{minipage}\tabularnewline
\begin{minipage}[t]{0.23\columnwidth}\raggedright\strut
\textbf{Autor}\strut
\end{minipage} & \begin{minipage}[t]{0.71\columnwidth}\raggedright\strut
Ignacio Zaldo González\strut
\end{minipage}\tabularnewline
\begin{minipage}[t]{0.23\columnwidth}\raggedright\strut
\textbf{Requisitos asociados}\strut
\end{minipage} & \begin{minipage}[t]{0.71\columnwidth}\raggedright\strut
RF-10.3\strut
\end{minipage}\tabularnewline
\begin{minipage}[t]{0.23\columnwidth}\raggedright\strut
\textbf{Descripción}\strut
\end{minipage} & \begin{minipage}[t]{0.71\columnwidth}\raggedright\strut
Permite al usuario con los roles necesarios editar la información de un jugador.\strut
\end{minipage}\tabularnewline
\begin{minipage}[t]{0.23\columnwidth}\raggedright\strut
\textbf{Precondición}\strut
\end{minipage} & \begin{minipage}[t]{0.71\columnwidth}\raggedright\strut
La sesión debe estar iniciada con un rol válido.\strut
\end{minipage}\tabularnewline
\begin{minipage}[t]{0.23\columnwidth}\raggedright\strut
\textbf{Acciones}\strut
\end{minipage} & \begin{minipage}[t]{0.71\columnwidth}\raggedright\strut
\begin{enumerate}
\def\labelenumi{\arabic{enumi}.}
\tightlist
\item
  El usuario se dirige a la ventana de jugadores.
\item
  El usuario pincha en el botón de editar jugador.
\item
  El sistema muestra el formulario con los datos que se pueden editar para ese jugador seleccionado.
\item
  El usuario edita la información.
\item
  El usuario pincha en el botón de guardar.
\item
  El sistema lanza la tarjeta de aviso de datos del jugador modificados.
\item
  El usuario puede ver el jugador con los datos nuevos en la lista.
\end{enumerate}\strut
\end{minipage}\tabularnewline
\begin{minipage}[t]{0.23\columnwidth}\raggedright\strut
\textbf{Postcondición}\strut
\end{minipage} & \begin{minipage}[t]{0.71\columnwidth}\raggedright\strut
Debe aparecer el jugador con los datos modificados en la lista de jugadores.\strut
\end{minipage}\tabularnewline
\begin{minipage}[t]{0.23\columnwidth}\raggedright\strut
\textbf{Excepciones}\strut
\end{minipage} & \begin{minipage}[t]{0.71\columnwidth}\raggedright\strut
\begin{itemize}
\tightlist
\item
  Error al modificar la información.
\item
  Datos modificados incorrectos.
\end{itemize}\strut
\end{minipage}\tabularnewline
\begin{minipage}[t]{0.23\columnwidth}\raggedright\strut
\textbf{Importancia}\strut
\end{minipage} & \begin{minipage}[t]{0.71\columnwidth}\raggedright\strut
Alta\strut
\end{minipage}\tabularnewline
\bottomrule
\caption{CU-17 Editar jugador.}
\end{longtable}

\newpage

\begin{longtable}[H]{@{}ll@{}}
\toprule
\begin{minipage}[b]{0.23\columnwidth}\raggedright\strut
\textbf{CU-18}\strut
\end{minipage} & \begin{minipage}[b]{0.71\columnwidth}\raggedright\strut
\textbf{Eliminar jugador}\strut
\end{minipage}\tabularnewline
\midrule
\endhead
\begin{minipage}[t]{0.23\columnwidth}\raggedright\strut
\textbf{Versión}\strut
\end{minipage} & \begin{minipage}[t]{0.71\columnwidth}\raggedright\strut
1.0\strut
\end{minipage}\tabularnewline
\begin{minipage}[t]{0.23\columnwidth}\raggedright\strut
\textbf{Autor}\strut
\end{minipage} & \begin{minipage}[t]{0.71\columnwidth}\raggedright\strut
Ignacio Zaldo González\strut
\end{minipage}\tabularnewline
\begin{minipage}[t]{0.23\columnwidth}\raggedright\strut
\textbf{Requisitos asociados}\strut
\end{minipage} & \begin{minipage}[t]{0.71\columnwidth}\raggedright\strut
RF-10.4\strut
\end{minipage}\tabularnewline
\begin{minipage}[t]{0.23\columnwidth}\raggedright\strut
\textbf{Descripción}\strut
\end{minipage} & \begin{minipage}[t]{0.71\columnwidth}\raggedright\strut
Permite al usuario con los roles necesarios eliminar un jugador.\strut
\end{minipage}\tabularnewline
\begin{minipage}[t]{0.23\columnwidth}\raggedright\strut
\textbf{Precondición}\strut
\end{minipage} & \begin{minipage}[t]{0.71\columnwidth}\raggedright\strut
La sesión debe estar iniciada con un rol válido.\strut
\end{minipage}\tabularnewline
\begin{minipage}[t]{0.23\columnwidth}\raggedright\strut
\textbf{Acciones}\strut
\end{minipage} & \begin{minipage}[t]{0.71\columnwidth}\raggedright\strut
\begin{enumerate}
\def\labelenumi{\arabic{enumi}.}
\tightlist
\item
  El usuario se dirige a la ventana de jugadores.
\item
  El usuario pincha en el botón de eliminar jugador.
\item
  El sistema muestra la modal de aviso de eliminación permanente de los datos.
\item
  El usuario pincha en el botón de aceptar.
\item
  El sistema lanza la tarjeta de aviso de datos del jugador eliminados.
\item
  El usuario ya no puede ver la información de ese jugador en la lista.
\end{enumerate}\strut
\end{minipage}\tabularnewline
\begin{minipage}[t]{0.23\columnwidth}\raggedright\strut
\textbf{Postcondición}\strut
\end{minipage} & \begin{minipage}[t]{0.71\columnwidth}\raggedright\strut
Debe desaparecer el jugador de la lista de jugadores.\strut
\end{minipage}\tabularnewline
\begin{minipage}[t]{0.23\columnwidth}\raggedright\strut
\textbf{Excepciones}\strut
\end{minipage} & \begin{minipage}[t]{0.71\columnwidth}\raggedright\strut
\begin{itemize}
\tightlist
\item
  Error al eliminar la información.
\end{itemize}\strut
\end{minipage}\tabularnewline
\begin{minipage}[t]{0.23\columnwidth}\raggedright\strut
\textbf{Importancia}\strut
\end{minipage} & \begin{minipage}[t]{0.71\columnwidth}\raggedright\strut
Alta\strut
\end{minipage}\tabularnewline
\bottomrule
\caption{CU-18 Eliminar jugador.}
\end{longtable}

\newpage

\begin{longtable}[H]{@{}ll@{}}
\toprule
\begin{minipage}[b]{0.23\columnwidth}\raggedright\strut
\textbf{CU-19}\strut
\end{minipage} & \begin{minipage}[b]{0.71\columnwidth}\raggedright\strut
\textbf{Gestión de partidos}\strut
\end{minipage}\tabularnewline
\midrule
\endhead
\begin{minipage}[t]{0.23\columnwidth}\raggedright\strut
\textbf{Versión}\strut
\end{minipage} & \begin{minipage}[t]{0.71\columnwidth}\raggedright\strut
1.0\strut
\end{minipage}\tabularnewline
\begin{minipage}[t]{0.23\columnwidth}\raggedright\strut
\textbf{Autor}\strut
\end{minipage} & \begin{minipage}[t]{0.71\columnwidth}\raggedright\strut
Ignacio Zaldo González\strut
\end{minipage}\tabularnewline
\begin{minipage}[t]{0.23\columnwidth}\raggedright\strut
\textbf{Requisitos asociados}\strut
\end{minipage} & \begin{minipage}[t]{0.71\columnwidth}\raggedright\strut
RF-11, RF-11.1, RF-11.2, RF-11.3, RF-11.4\strut
\end{minipage}\tabularnewline
\begin{minipage}[t]{0.23\columnwidth}\raggedright\strut
\textbf{Descripción}\strut
\end{minipage} & \begin{minipage}[t]{0.71\columnwidth}\raggedright\strut
Permite al usuario con el rol necesario gestionar la información de los partidos.\strut
\end{minipage}\tabularnewline
\begin{minipage}[t]{0.23\columnwidth}\raggedright\strut
\textbf{Precondición}\strut
\end{minipage} & \begin{minipage}[t]{0.71\columnwidth}\raggedright\strut
Debe haber partidos creados.\strut
\end{minipage}\tabularnewline
\begin{minipage}[t]{0.23\columnwidth}\raggedright\strut
\textbf{Acciones}\strut
\end{minipage} & \begin{minipage}[t]{0.71\columnwidth}\raggedright\strut
\begin{enumerate}
\def\labelenumi{\arabic{enumi}.}
\tightlist
\item
  El usuario entra en la aplicación.
\item
  El usuario se dirige a la ventana de partidos.
\item
  El sistema devuelve la lista de todos los partidos.
\item
  El usuario elige que acción realizar.
\end{enumerate}\strut
\end{minipage}\tabularnewline
\begin{minipage}[t]{0.23\columnwidth}\raggedright\strut
\textbf{Postcondición}\strut
\end{minipage} & \begin{minipage}[t]{0.71\columnwidth}\raggedright\strut
La lista de partidos se muestra correctamente.\strut
\end{minipage}\tabularnewline
\begin{minipage}[t]{0.23\columnwidth}\raggedright\strut
\textbf{Excepciones}\strut
\end{minipage} & \begin{minipage}[t]{0.71\columnwidth}\raggedright\strut
\begin{itemize}
\tightlist
\item
  Error al cargar la lista.
\item
  Se ha pinchado en una ventana diferente.
\end{itemize}\strut
\end{minipage}\tabularnewline
\begin{minipage}[t]{0.23\columnwidth}\raggedright\strut
\textbf{Importancia}\strut
\end{minipage} & \begin{minipage}[t]{0.71\columnwidth}\raggedright\strut
Alta\strut
\end{minipage}\tabularnewline
\bottomrule
\caption{CU-19 Gestión de partidos.}
\end{longtable}

\newpage

\begin{longtable}[H]{@{}ll@{}}
\toprule
\begin{minipage}[b]{0.23\columnwidth}\raggedright\strut
\textbf{CU-20}\strut
\end{minipage} & \begin{minipage}[b]{0.71\columnwidth}\raggedright\strut
\textbf{Ver partido}\strut
\end{minipage}\tabularnewline
\midrule
\endhead
\begin{minipage}[t]{0.23\columnwidth}\raggedright\strut
\textbf{Versión}\strut
\end{minipage} & \begin{minipage}[t]{0.71\columnwidth}\raggedright\strut
1.0\strut
\end{minipage}\tabularnewline
\begin{minipage}[t]{0.23\columnwidth}\raggedright\strut
\textbf{Autor}\strut
\end{minipage} & \begin{minipage}[t]{0.71\columnwidth}\raggedright\strut
Ignacio Zaldo González\strut
\end{minipage}\tabularnewline
\begin{minipage}[t]{0.23\columnwidth}\raggedright\strut
\textbf{Requisitos asociados}\strut
\end{minipage} & \begin{minipage}[t]{0.71\columnwidth}\raggedright\strut
RF-11.1\strut
\end{minipage}\tabularnewline
\begin{minipage}[t]{0.23\columnwidth}\raggedright\strut
\textbf{Descripción}\strut
\end{minipage} & \begin{minipage}[t]{0.71\columnwidth}\raggedright\strut
Permite al usuario visualizar la información de cada partido.\strut
\end{minipage}\tabularnewline
\begin{minipage}[t]{0.23\columnwidth}\raggedright\strut
\textbf{Precondición}\strut
\end{minipage} & \begin{minipage}[t]{0.71\columnwidth}\raggedright\strut
Debe haber partido creados.\strut
\end{minipage}\tabularnewline
\begin{minipage}[t]{0.23\columnwidth}\raggedright\strut
\textbf{Acciones}\strut
\end{minipage} & \begin{minipage}[t]{0.71\columnwidth}\raggedright\strut
\begin{enumerate}
\def\labelenumi{\arabic{enumi}.}
\tightlist
\item
  El usuario se dirige a la ventana de partidos.
\item
  El usuario pincha en la fecha del partido para expandir su información.
\item
  El sistema devuelve la tarjeta de información del partido.
\item
  El usuario visualiza la información.
\end{enumerate}\strut
\end{minipage}\tabularnewline
\begin{minipage}[t]{0.23\columnwidth}\raggedright\strut
\textbf{Postcondición}\strut
\end{minipage} & \begin{minipage}[t]{0.71\columnwidth}\raggedright\strut
La información detallada del partido se muestra correctamente.\strut
\end{minipage}\tabularnewline
\begin{minipage}[t]{0.23\columnwidth}\raggedright\strut
\textbf{Excepciones}\strut
\end{minipage} & \begin{minipage}[t]{0.71\columnwidth}\raggedright\strut
\begin{itemize}
\tightlist
\item
  Error al cargar la información.
\item
  No existe el partido.
\end{itemize}\strut
\end{minipage}\tabularnewline
\begin{minipage}[t]{0.23\columnwidth}\raggedright\strut
\textbf{Importancia}\strut
\end{minipage} & \begin{minipage}[t]{0.71\columnwidth}\raggedright\strut
Alta\strut
\end{minipage}\tabularnewline
\bottomrule
\caption{CU-20 Ver partido.}
\end{longtable}

\newpage

\begin{longtable}[H]{@{}ll@{}}
\toprule
\begin{minipage}[b]{0.23\columnwidth}\raggedright\strut
\textbf{CU-21}\strut
\end{minipage} & \begin{minipage}[b]{0.71\columnwidth}\raggedright\strut
\textbf{Añadir partido}\strut
\end{minipage}\tabularnewline
\midrule
\endhead
\begin{minipage}[t]{0.23\columnwidth}\raggedright\strut
\textbf{Versión}\strut
\end{minipage} & \begin{minipage}[t]{0.71\columnwidth}\raggedright\strut
1.0\strut
\end{minipage}\tabularnewline
\begin{minipage}[t]{0.23\columnwidth}\raggedright\strut
\textbf{Autor}\strut
\end{minipage} & \begin{minipage}[t]{0.71\columnwidth}\raggedright\strut
Ignacio Zaldo González\strut
\end{minipage}\tabularnewline
\begin{minipage}[t]{0.23\columnwidth}\raggedright\strut
\textbf{Requisitos asociados}\strut
\end{minipage} & \begin{minipage}[t]{0.71\columnwidth}\raggedright\strut
RF-11.2\strut
\end{minipage}\tabularnewline
\begin{minipage}[t]{0.23\columnwidth}\raggedright\strut
\textbf{Descripción}\strut
\end{minipage} & \begin{minipage}[t]{0.71\columnwidth}\raggedright\strut
Permite al usuario con los roles necesarios añadir un nuevo partido.\strut
\end{minipage}\tabularnewline
\begin{minipage}[t]{0.23\columnwidth}\raggedright\strut
\textbf{Precondición}\strut
\end{minipage} & \begin{minipage}[t]{0.71\columnwidth}\raggedright\strut
La sesión debe estar iniciada con un rol válido.\strut
\end{minipage}\tabularnewline
\begin{minipage}[t]{0.23\columnwidth}\raggedright\strut
\textbf{Acciones}\strut
\end{minipage} & \begin{minipage}[t]{0.71\columnwidth}\raggedright\strut
\begin{enumerate}
\def\labelenumi{\arabic{enumi}.}
\tightlist
\item
  El usuario se dirige a la ventana de partidos.
\item
  El usuario pincha en el botón de añadir partido.
\item
  El sistema muestra el formulario con los datos necesarios para añadir el nuevo partido.
\item
  El usuario introduce la información.
\item
  El usuario pincha en el botón de guardar.
\item
  El sistema lanza la tarjeta de aviso de nuevo partido agregado.
\item
  El usuario puede ver el nuevo partido en la lista.
\end{enumerate}\strut
\end{minipage}\tabularnewline
\begin{minipage}[t]{0.23\columnwidth}\raggedright\strut
\textbf{Postcondición}\strut
\end{minipage} & \begin{minipage}[t]{0.71\columnwidth}\raggedright\strut
Debe aparecer el nuevo partido añadido en la lista de partidos.\strut
\end{minipage}\tabularnewline
\begin{minipage}[t]{0.23\columnwidth}\raggedright\strut
\textbf{Excepciones}\strut
\end{minipage} & \begin{minipage}[t]{0.71\columnwidth}\raggedright\strut
\begin{itemize}
\tightlist
\item
  Error al guardar la información.
\item
  Datos introducidos incorrectos.
\end{itemize}\strut
\end{minipage}\tabularnewline
\begin{minipage}[t]{0.23\columnwidth}\raggedright\strut
\textbf{Importancia}\strut
\end{minipage} & \begin{minipage}[t]{0.71\columnwidth}\raggedright\strut
Alta\strut
\end{minipage}\tabularnewline
\bottomrule
\caption{CU-21 Añadir partido.}
\end{longtable}

\newpage

\begin{longtable}[H]{@{}ll@{}}
\toprule
\begin{minipage}[b]{0.23\columnwidth}\raggedright\strut
\textbf{CU-22}\strut
\end{minipage} & \begin{minipage}[b]{0.71\columnwidth}\raggedright\strut
\textbf{Editar partido}\strut
\end{minipage}\tabularnewline
\midrule
\endhead
\begin{minipage}[t]{0.23\columnwidth}\raggedright\strut
\textbf{Versión}\strut
\end{minipage} & \begin{minipage}[t]{0.71\columnwidth}\raggedright\strut
1.0\strut
\end{minipage}\tabularnewline
\begin{minipage}[t]{0.23\columnwidth}\raggedright\strut
\textbf{Autor}\strut
\end{minipage} & \begin{minipage}[t]{0.71\columnwidth}\raggedright\strut
Ignacio Zaldo González\strut
\end{minipage}\tabularnewline
\begin{minipage}[t]{0.23\columnwidth}\raggedright\strut
\textbf{Requisitos asociados}\strut
\end{minipage} & \begin{minipage}[t]{0.71\columnwidth}\raggedright\strut
RF-11.3\strut
\end{minipage}\tabularnewline
\begin{minipage}[t]{0.23\columnwidth}\raggedright\strut
\textbf{Descripción}\strut
\end{minipage} & \begin{minipage}[t]{0.71\columnwidth}\raggedright\strut
Permite al usuario con los roles necesarios editar la información de un partido.\strut
\end{minipage}\tabularnewline
\begin{minipage}[t]{0.23\columnwidth}\raggedright\strut
\textbf{Precondición}\strut
\end{minipage} & \begin{minipage}[t]{0.71\columnwidth}\raggedright\strut
La sesión debe estar iniciada con un rol válido.\strut
\end{minipage}\tabularnewline
\begin{minipage}[t]{0.23\columnwidth}\raggedright\strut
\textbf{Acciones}\strut
\end{minipage} & \begin{minipage}[t]{0.71\columnwidth}\raggedright\strut
\begin{enumerate}
\def\labelenumi{\arabic{enumi}.}
\tightlist
\item
  El usuario se dirige a la ventana de partidos.
\item
  El usuario pincha en el botón de editar partido.
\item
  El sistema muestra el formulario con los datos que se pueden editar para ese partido seleccionado.
\item
  El usuario edita la información.
\item
  El usuario pincha en el botón de guardar.
\item
  El sistema lanza la tarjeta de aviso de datos del partido modificados.
\item
  El usuario puede ver el partido con los datos nuevos en la lista.
\end{enumerate}\strut
\end{minipage}\tabularnewline
\begin{minipage}[t]{0.23\columnwidth}\raggedright\strut
\textbf{Postcondición}\strut
\end{minipage} & \begin{minipage}[t]{0.71\columnwidth}\raggedright\strut
Debe aparecer el partido con los datos modificados en la lista de partidos.\strut
\end{minipage}\tabularnewline
\begin{minipage}[t]{0.23\columnwidth}\raggedright\strut
\textbf{Excepciones}\strut
\end{minipage} & \begin{minipage}[t]{0.71\columnwidth}\raggedright\strut
\begin{itemize}
\tightlist
\item
  Error al modificar la información.
\item
  Datos modificados incorrectos.
\end{itemize}\strut
\end{minipage}\tabularnewline
\begin{minipage}[t]{0.23\columnwidth}\raggedright\strut
\textbf{Importancia}\strut
\end{minipage} & \begin{minipage}[t]{0.71\columnwidth}\raggedright\strut
Alta\strut
\end{minipage}\tabularnewline
\bottomrule
\caption{CU-22 Editar partido.}
\end{longtable}

\newpage

\begin{longtable}[H]{@{}ll@{}}
\toprule
\begin{minipage}[b]{0.23\columnwidth}\raggedright\strut
\textbf{CU-23}\strut
\end{minipage} & \begin{minipage}[b]{0.71\columnwidth}\raggedright\strut
\textbf{Eliminar partido}\strut
\end{minipage}\tabularnewline
\midrule
\endhead
\begin{minipage}[t]{0.23\columnwidth}\raggedright\strut
\textbf{Versión}\strut
\end{minipage} & \begin{minipage}[t]{0.71\columnwidth}\raggedright\strut
1.0\strut
\end{minipage}\tabularnewline
\begin{minipage}[t]{0.23\columnwidth}\raggedright\strut
\textbf{Autor}\strut
\end{minipage} & \begin{minipage}[t]{0.71\columnwidth}\raggedright\strut
Ignacio Zaldo González\strut
\end{minipage}\tabularnewline
\begin{minipage}[t]{0.23\columnwidth}\raggedright\strut
\textbf{Requisitos asociados}\strut
\end{minipage} & \begin{minipage}[t]{0.71\columnwidth}\raggedright\strut
RF-11.4\strut
\end{minipage}\tabularnewline
\begin{minipage}[t]{0.23\columnwidth}\raggedright\strut
\textbf{Descripción}\strut
\end{minipage} & \begin{minipage}[t]{0.71\columnwidth}\raggedright\strut
Permite al usuario con los roles necesarios eliminar un partido.\strut
\end{minipage}\tabularnewline
\begin{minipage}[t]{0.23\columnwidth}\raggedright\strut
\textbf{Precondición}\strut
\end{minipage} & \begin{minipage}[t]{0.71\columnwidth}\raggedright\strut
La sesión debe estar iniciada con un rol válido.\strut
\end{minipage}\tabularnewline
\begin{minipage}[t]{0.23\columnwidth}\raggedright\strut
\textbf{Acciones}\strut
\end{minipage} & \begin{minipage}[t]{0.71\columnwidth}\raggedright\strut
\begin{enumerate}
\def\labelenumi{\arabic{enumi}.}
\tightlist
\item
  El usuario se dirige a la ventana de partidos.
\item
  El usuario pincha en el botón de eliminar partido.
\item
  El sistema muestra la modal de aviso de eliminación permanente de los datos.
\item
  El usuario pincha en el botón de aceptar.
\item
  El sistema lanza la tarjeta de aviso de datos del partido eliminados.
\item
  El usuario ya no puede ver la información de ese partido en la lista.
\end{enumerate}\strut
\end{minipage}\tabularnewline
\begin{minipage}[t]{0.23\columnwidth}\raggedright\strut
\textbf{Postcondición}\strut
\end{minipage} & \begin{minipage}[t]{0.71\columnwidth}\raggedright\strut
Debe desaparecer el partido de la lista de partidos.\strut
\end{minipage}\tabularnewline
\begin{minipage}[t]{0.23\columnwidth}\raggedright\strut
\textbf{Excepciones}\strut
\end{minipage} & \begin{minipage}[t]{0.71\columnwidth}\raggedright\strut
\begin{itemize}
\tightlist
\item
  Error al eliminar la información.
\end{itemize}\strut
\end{minipage}\tabularnewline
\begin{minipage}[t]{0.23\columnwidth}\raggedright\strut
\textbf{Importancia}\strut
\end{minipage} & \begin{minipage}[t]{0.71\columnwidth}\raggedright\strut
Alta\strut
\end{minipage}\tabularnewline
\bottomrule
\caption{CU-23 Eliminar partido.}
\end{longtable}