\capitulo{7}{Conclusiones y Líneas de trabajo futuras}

En este apartado se exponen todas las conclusiones del proyecto y además, las posibles líneas de trabajo futuras que pueden dar continuidad al proyecto.

\section{Conclusiones}\label{conclusiones}

Después de haber realizado todo el desarrollo del trabajo, las conclusiones que saco son:

\begin{itemize}
    \item El proyecto me ha resultado muy útil, ya que he adquirido muchos conocimientos nuevos que no tenía antes de comenzar con el desarrollo y he cumplido los objetivos que tenía, que están comentados en las secciones anteriores. El objetivo inicial se ha cumplido y además por el camino han ido surgiendo nuevas ideas que se han ido añadiendo como complementos que son útiles y aportan valor a la aplicación.
    \item Desde el punto de vista más personal, el haber trabajado en un proyecto que implicaba un deporte que me gusta y usar tecnologías nuevas que no conocía me han hecho desarrollarlo con más facilidad y motivación. Además, el punto de integrar un chat inteligente con inteligencia artificial dentro de la web, también ha sido interesante y me ha hecho aprender también sobre este tema.
    \item En cuanto a las tecnologías utilizadas para llevar a cabo el proyecto, algunas de ellas han resultado más interesantes que otras, en especial las de desarrollo de código. La herramienta que se ha empleado para hacer la documentación ha sido LaTeX y si que ha implicado más horas de las previstas al tener que adaptarse a otra nueva herramienta.
\end{itemize}

En resumen, se podría decir que la realización del proyecto ha sido una experiencia satisfactoria y ha resultado interesante sobretodo por la motivación que suponía el uso de nuevas tecnologías y implicar temas más llevaderos y amigables en el proyecto como es el tenis. Las competencias que se han adquirido con el desarrollo son muy valiosas para continuar con la carrera profesional.


\section{Líneas de trabajo futuras}\label{lineas-de-trabajo-futuras}

A pesar de que el proyecto está completo y es perfectamente funcional, se podrían añadir más mejoras y funcionalidades que podrían hacer mejor la experiencia del usuario final de la aplicación.

A continuación, se proponen algunas líneas de trabajo que se podrían seguir para continuar con el proyecto y sus mejoras:

\begin{itemize}
    \item Mejora de la seguridad de la aplicación: la seguridad de los datos es algo importante y aunque en la aplicación se valida que si un usuario no tiene el rol necesario para modificar o añadir datos, siempre se puede mejorar y añadir algo de extra de seguridad como por ejemplo validando la API que maneja los datos.
    \item Añadir más relaciones y datos interesantes: se pueden añadir más estadísticas o relaciones entre datos para que el usuario que utilice la aplicación pueda ver más información en una misma ventana sin tener que cambiar a otra diferente.
    \item Mejorar la implementación del chatbot: el chatbot, aunque ya está muy avanzado y se adapta a las preguntas que hace el usuario, se puede mejorar integrandolo en otros lenguajes de programación más adaptados a inteligencia artificial como es python y así evitar el proceso de reconocimiento de cadenas de texto. Además, la ia también va avanzando y gemini que es el modelo que se utiliza en este caso (en su versión 1.5), también añadirá nuevas funcionalidades y mejores capacidades.
\end{itemize}