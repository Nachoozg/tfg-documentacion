\capitulo{2}{Objetivos del proyecto}

En este apartado se van a detallar los objetivos que se quieren alcanzar realizando este proyecto, haciendo una diferenciación entre los objetivos generales, que están relacionados con los requisitos que va a cumplir el software, objetivos técnicos, que son los aprendizajes que necesitaré para alcanzar los objetivos generales y los objetivos personales, relacionados con el punto final al que quiero llegar con este proyecto.

\section{Objetivos generales}\label{objetivos-generales}

\begin{itemize}
\tightlist
\item	
\textbf{Desarrollar la aplicación web completa:} Implementar toda la lógica de programación necesaria para llegar al producto software final.

\item
\textbf{Desarrollar la parte front-end de la web:} La aplicación es fullstack, por tanto, el front-end vendrá desarrollado en Angular que permitirá aprovechar su estructura para que la aplicación sea escalable fácilmente.

\item
\textbf{Desarrollar la parte back-end de la web:} El back-end vendrá desarrollado en .Net, permitiendo desarrollar APIs propias con las que alterar información de bases de datos. Además, el desarrollo de las tablas de la base de datos se llevará a cabo usando HeidiSQL.

\item
\textbf{Sistema de autenticación:} Implementar un sistema que permita validar credenciales de usuarios con diferentes roles que permitirán ver o alterar una información diferente en función del rol asignado. De esta manera, el mantenimiento de los datos será seguro, ya que nadie que no tenga el rol indicado podrá acceder a ella y modificarla.

\item
\textbf{Mostrar información necesaria sobre la liga de tenis:} Proporcionar información relevante sobre los colegios que participan, los jugadores que hay en cada colegio compitiendo y estadísticas de clasificación general, así como información necesaria sobre los partidos jugados y por jugar entre los jugadores representantes de los diferentes colegios.

\item
\textbf{Generar predicciones sobre los partidos por jugar:} Agregar una predicción de resultados de los partidos que se van a jugar más adelante mediante la evaluación de estadísticas de anteriores encuentros entre los colegios y últimos encuentros disputados.
\end{itemize}




\section{Objetivos técnicos}\label{objetivos-tecnicos}

\begin{itemize}
\tightlist
\item
\textbf{Implementar Bootstrap:} para mejorar la apariencia y la forma en la que se navega en la página web. Al implementar esta herramienta, permite que la web sea responsive, es decir, tenga un diseño que se adapte al dispositivo en el que se está viendo y la funcionalidad cambie en los menús y barras de navegación, por ejemplo.

\item
\textbf{Implementar Angular Material:} Permite añadir barras de navegación y menús con una estructura ya definida y que evitan tener que partir de cero al hacer estos componentes.

\item
\textbf{Implementar Toastr:} Permite mostrar alertas que tienen una estructura y permiten ser llamadas en el código de una forma definida, con la que no hace falta crear las tarjetas de alerta desde 0 y hacer su funcionalidad y su diseño.

\item
\textbf{Implementar Full Calendar:} Permite utilizar el diseño y la funcionalidad de un calendario, al que se le pueden añadir eventos y botones como el cambio de mes sin tener que hacer un calendario con sus diferentes funcionalidades desde cero.

\item
\textbf{Implementar Leaflet y OpenStreetMap:} Implementar esta librería (Leaflet) y esta API gratuita (OpenStreetMap) permite mostrar un mapa sobre el cual se puede indicar la ubicación en la que se juega un partido a través de una interacción directa con el mapa que se muestra en la interfaz del usuario.
\end{itemize}


\section{Objetivos personales}\label{objetivos-personales}

\begin{itemize}
\tightlist
\item
\textbf{Mejorar el nivel en el desarrollo del front-end con Angular:} Conseguir mucha experiencia desarrollando aplicaciones web y utilizar al mejor nivel que pueda todas las posibilidades y opciones que ofrece angular respecto a otros frameworks.

\item
\textbf{Mejorar el nivel en el desarrollo del back-end con .Net para el desarrollo de APIs:} Conseguir experiencia y desarrollar conocimientos avanzados sobre el desarrollo de APIs para poder manejar mis propios datos de forma segura y controlada.

\item
\textbf{Ampliar mi capacidad de diseñar interfaces de usuario:} Conseguir mejoras significativas en mi forma de desenvolverme a la hora de crear interfaces de usuario intuitivas y sencillas de usar para el usuario final que vaya a utilizar la aplicación, consiguiendo crear diseños atractivos y útiles.

\item
\textbf{Ampliar mis conocimientos sobre bases de datos:} Mejorar mis conocimientos sobre el uso de bases de datos en este caso mediante el uso de MySQL a través de HeidiSQL que permite crear tablas para bases de datos de una forma más visual, intuitiva y fácil.

\item
\textbf{Aprender a integrar una buena seguridad en los datos que hay detrás de una aplicación:} Conseguir adquirir buenos conocimientos sobre la seguridad de los datos en aplicaciones reales y sobre cómo se lleva a cabo el manejo de ellos sin filtrar datos privados o que ciertos tipos de usuarios no deberían de poder visualizar.
\end{itemize}