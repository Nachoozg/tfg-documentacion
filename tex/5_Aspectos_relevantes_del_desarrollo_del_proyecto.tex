\capitulo{5}{Aspectos relevantes del desarrollo del proyecto}

En este apartado se van a presentar los aspectos más importantes de todo el desarrollo del proyecto. Se incluye desde el inicio del proyecto y la formación realizada hasta todo el proceso de desarrollo del software, el diseño y la resolución de problemas que se han ido planteando con el tiempo.

\section{Inicio del proyecto}\label{inicio-del-proyecto}

La idea de este proyecto apareció como la idea de combinar uno de los deportes que más he practicado y más disfruto viéndolo con mis estudios y la programación.

El tenis es un deporte que he practicado durante varios años de mi vida y siempre me ha gustado ver partidos de competiciones importantes para ver a los mejores en ello.

En concreto, escogí realizar una web para la gestión de una liga de tenis para los colegios de Burgos porque es algo que no existe y es novedoso. En las instalaciones de la escuela en la que iba yo a jugar a tenis había tablones en las paredes en los que se colgaban papeles con tablas de jugadores que estaban inscritos en torneos que se jugarían más adelante. Todo estaba gestionado con los tablones de anuncios y si quería consultar cuando tenía que jugar un partido o quienes estaban inscritos para jugar el torneo, tenía que acercarme hasta el lugar a consultar el papel.

Además, en el caso de el colegio al que yo iba, no había clases de tenis y por tanto, mucho menos había competición o partidos contra otras personas. 

Esa necesidad que vi, junto con el interés que me despertó el desarrollo web cuando lo descubrí, me hicieron que utilizar las herramientas que he usado para el desarrollo (Angular y .NET) me resultara más interesante y más llevadero para aprenderlo.

Una vez llegado a la idea y tras recibir el visto bueno del tutor, me puse con el desarrollo de todo el proyecto.

\section{Formación}\label{formacion}

Las herramientas que se utilizan para desarrollar este proyecto no son herramientas que se hayan visto en las asignaturas de la carrera o se ha tocado muy poco o sin relación al desarrollo web, por tanto, se requerían conocimientos de los que no disponía al comienzo del desarrollo. Para adquirir esos conocimientos que no tenía me puse a investigar en profundidad acerca de estas tecnologías y tuve que realizar cursos formativos con mucho material que me fuera útil.

Además, con el auge de la inteligencia artificial en los últimos años, finalmente surgió la idea de añadir un asistente inteligente que aparecería en forma de chat en la web y también implicaba investigación y adquirir nuevos conocimientos sobre como llevar la idea a cabo.

\subsection{Angular}

Angular era nuevo para mí, sabía que existían frameworks como React y Django pero no Angular. Realicé un par de cursos que me fueron útiles para poder aprender mucho sobre el tema:

\begin{itemize}
\tightlist
\item
  \emph{Legacy - Angular: De cero a experto} (Udemy)
  \cite{course:angular_legacy}.
\item
  \emph{Aprende ANGULAR 17 desde cero para principiantes GRATIS} (Youtube)
  \cite{course:angular_midudev}.
\item
  \emph{Documentación de Angular en español}
  \cite{web:angular_hispano}.
\item
  \emph{Documentación oficial de Angular en inglés}
  \cite{web:angular_oficial}.
\end{itemize}


\subsection{.NET}

Sobre .NET ya tenía algo de conocimiento y además, el lenguaje es similar a Java, que si que se estudia en la carrera y por tanto, partía de algo de experiencia utilizándolo. Aún así, también consulté algunos tutoriales para conseguir obtener más información y experiencia con esta herramienta.

\begin{itemize}
\tightlist
\item
  \emph{Curso .NET Core} (Youtube)
  \cite{web:curso-net}.
\item
  \emph{.NET 8 Web API and Entity Framework} (Youtube)
  \cite{web:net-web-api}.
\item
  \emph{Paseo por el lenguaje C sharp}
  \cite{web:net_tutorial}.
\item
  \emph{Documentación de .NET}
  \cite{web:net_documentacion}.
\item
  \emph{documentación de ASP.NET}
  \cite{web:net-asp}.
\end{itemize}


Además, para todos los problemas que han ido surgiendo a lo largo del desarrollo, también ha sido muy útil recurrir a páginas como \href{https://stackoverflow.com/questions}{StackOverflow} y \href{https://github.com/community}{Github Community}, donde se muestran los mismos problemas que han tenido otros usuarios y otros usuarios responden con las posibles respuestas a los problemas planteados.

\section{Desarrollo}\label{desarrollo}

En esta sección se van a describir los aspectos más relevantes de todo el proceso de desarrollo del proyecto. Todos los aspectos señalados han sido puntos clave para poder conseguir este producto final.

\subsection{Creación del front-end de la web}

Lo primero que se hizo al comenzar con el desarrollo de este trabajo fue crear el proyecto en Angular y comenzar el desarrollo de toda la estructura básica para poder tener el esqueleto de la web creado, con esto ya se podían ver las secciones principales del proyecto en un inicio, la barra de navegación entre las diferentes partes de la web y toda la estructura de formularios y botones conectados entre ellos para después poder comenzar con el desarrollo de la API en back-end y poder hacer operaciones sobre la base de datos.

\imagen{listadoInicial.png}{Estructura inicial sin conexión a base de datos}{1.0}

\imagen{formInicial.png}{Formulario creado inicialmente (Vista del formulario de colegios)}{1.0}

\subsection{Creación de la base de datos}

Antes de pasar a crear la API se creó la base de datos en HeidiSQL con sus diferentes tablas que se necesitarían en un principio y que después con el tiempo fueron evolucionando a tener más columnas o tablas nuevas para almacenar datos nuevos. Con la creación anterior de los formularios al hacer la estructura básica de la web en Angular, ya se tenía el conocimiento necesario para saber que datos eran necesarios para poder tener todo conectado y poder almacenar toda la información que sería necesaria para llevar la correcta gestión de la liga.


\section{Integración de APIs}\label{integracion-apis}
\subsection{Creación e integración de APIs propias}

Uno de los puntos principales del desarrollo fue la creación de la API propia para poder sacar, meter, modificar y consultar datos necesarios de las tablas de la base de datos y así poder tener las tablas para ver la información y los formularios para editar y añadir esa información. Esta API se creó en .NET con el tipo de solución que se ve en la imagen.

\imagen{tipo-proyecto.jpg}{Tipo de proyecto elegido}{0.7}

A medida que se iban creando diferentes endpoints, se iban probando en Swagger para ver si funcionaban correctamente o si de lo contrario fallaba algo y había que arreglarlo. Cuando esos endpoints estaban creados y probados, se creaba la conexión del back con el front a través de los endpoints que se habían creado.

\subsection{Integración de APIs externas}

Otro aspecto relevante fue la integracion del proyecto con APIs externas, que en este caso se utilizaron dos y fueron:

\subsubsection{Leaflet}

Para poder añadir una modal en la web en la que se mostraría un mapa interactivo para ver donde se jugarían los partidos. En el caso de ser un usuario normal, mostraría la ubicación del partido únicamente, pero en caso de ser un usuario con roles de administrador o arbitro, permitiría poner el puntero en el mapa para marcar en que lugar se disputaría el partido que se estaba creando o editando.

\subsubsection{Google AI}

La integración con esta API permitió añadir un chatbot inteligente a la web para poder hacerle consultas sobre temas relacionados con la liga de tenis y no tener que ir manualmente a consultar las diferentes tablas que estaban en las secciones específicas. Para seleccionar esta API, se eligió entre otras (como por ejemplo, la de OpenAI) porque a la hora de usar la web de forma real, en caso de que se utilizara para gestionar una liga real de tenis para colegios, estas APIs tienen unos costes por consulta y por número de tokens usados en entornos de producción, y google ai studio es la que ofrece menores costes y mejor desempeño y facilidades a la hora de integrarlo en nuestras webs propias.


\section{Diseño}\label{diseño}

El diseño de la interfaz de usuario de la web, también fue uno de los aspectos clave para poder tener una web amigable y que fuera sobre todo intuitiva y fácil de utilizar para el usuario final. Para ello, en un inicio, el diseño era muy simple y se tenía sin colores específicos y con una distribución de los elementos básica solamente con el diseño por defecto al ir creando los componentes y servicios.

\imagen{form-jugador.png}{Formulario inicial para agregar jugadores}{1.0}

Cuando se llegó al punto en el que todo el front estaba perfectamente integrado con el back y se podían hacer todas las operaciones necesarias para gestionar los datos de la web, se comenzó a mejorar la interfaz de usuario añadiendo estilos, colores, tipografía y creando una distribución de los elementos, botones, tablas, imágenes, etc. que permitiera facilitar lo máximo posible el uso de la web.

Estas imágenes muestran algunas ventanas de la aplicación antes y después de empezar con las mejoras de diseño.

\imagen{vista-jugador.png}{Vista de la información del jugador antes de comenzar la mejora de diseño}{0.6}

\imagen{tabla-clasificacion.png}{Vista de la tabla de clasificación sin mejoras en su diseño}{0.6}

\imagen{calendario-inicial.png}{Vista del calendario creado al mejorar el diseño}{0.5}

\imagen{lista-colegios.png}{Listado de colegios al mejorar el diseño}{0.6}

Para ello fue importante ponerse en la posición del usuario final y navegar por la web sin saber como se utilizaba para llegar a conclusiones de las que se sacaría un rediseño nuevo para llegar al producto final. Para lograr este diseño final, se han utilizado varias herramientas y tecnologías de diseño como Bootstrap y Angular Material. Con Bootstrap se ha hecho una gestión de los estilos y creación de botones y barra de navegación. Con Angular Material también se han creado componentes de la web como botones y tablas. Para poder mostrar alertas al usuario y darle advertencias sobre los datos que estaba añadiendo, editando o eliminando, se utilizó toastr, otra librería que permite mostrar alertas emergentes con información específica y con una duración específica.
En los puntos finales del desarrollo, también se ha adaptado el diseño para el uso de la web en dispositivos móviles, haciendo que el diseño sea responsive y se compriman los elementos al tamaño de la pantalla utilizada.

En las imagenes siguientes se muestra el diseño final de las partes de la web mostradas anteriormente en fases de desarrollo:


\imagen{disenoInicio.png}{Diseño final de la página principal de la aplicación}{1.0}

\imagen{vistaJugadorFinal.png}{Vista final de la información del jugador}{1.0}

\imagen{tablaClasFinal.png}{Vista de la tabla de clasificación final}{1.0}

\imagen{listadoColFinal.png}{Listado de colegios final}{1.0}

\imagen{vista-chat.png}{Diseño del chatbot inteligente integrado}{0.3}

Para la creación de las imágenes como el logotipo de la aplicación se ha utilizado la herramienta paint, que permite hacer imágenes básicas como son las de la imagen de la ventana de la web y la pestaña de inicio.
Las imágenes que se han utilizado en la web son:
\begin{itemize}
\tightlist
\item
  \emph{Imágen raqueta de tenis} (Unsplash)
  \cite{img:raqueta-img}.
\item
  \emph{Imágen de hombre jugando a tenis} (Unsplash)
  \cite{img:hombre-tenis}.
\end{itemize}

\section{Resolución de problemas}\label{resolucion-problemas}

Durante el desarrollo del proyecto se han ido presentando varios problemas o desafíos que se han tenido que ir resolviendo para lograr el funcionamiento correcto de todo.

\subsection{Problema de conexión del front-end con el back-end}

El primer problema que surgió fue como hacer bien la conexión entre Angular y .NET para poder manejar los datos de las tablas de la base de datos y hacer las conexiones correctamente. 

Para resolver el problema se tuvieron que crear servicios de Angular que apuntaran a las url generadas por los endpoints que se habían creado en .NET.

\imagen{url-endpoints.png}{Ejemplo de url de conexión con los endpoints desde un servicio en Angular}{0.8}

\subsection{Problema con la subida y visualización de imágenes}

La integración de la posibilidad de meter imágenes en los formularios para los jugadores y los colegios hizo que surgiera el problema de que al subir una imagen, se paraba la ejecución de todo el proyecto automáticamente y daba errores. Ese problema venía porque se estaban metiendo directamente las imágenes a la base de datos y esto hacía que colapsara todo.

Para resolver este problema lo que se hizo fue hacer modificaciones en .NET y configurarlo de forma que las imágenes que se subieran se guardarían en un servidor interno de.NET (wwwroot) y lo que se guardaría en base de datos era solo el enlace a esa imágen para evitar subir mucho peso a la base de datos.


\imagen{wwwroot.png}{Estructura de wwwroot en .NET}{0.6}

\subsection{Problema con los estilos de FullCalendar}

Estilizar el calendario no fue fácil ya que FullCalendar presenta su diseño por defecto y hay configuraciones para cambiar un poco los estilos de los botones y cambiar el idioma por defecto del calendario que es el inglés.

Para dar con la solución se tuvo que investigar durante un tiempo en foros y páginas que aportaban distintas soluciones e ir adaptando poco a poco el diseño para conseguir lo que se quería.

\subsection{Problema con la subida del proyecto a Amazon Web Services}

Antes de hacer este proyecto partía sin idea de como se podía hacer para que la web funcionara directamente buscando su url en cualquier dispositivo en vez de ejecutar la aplicación en local desde mi ordenador.

Para resolver esto y poder cumplir con ello, este vídeo sirvió de ayuda \href{https://www.youtube.com/watch?v=nQdyiK7-VlQ}{Easily deploy full stack app} junto con muchas consultas a foros como StackOverflow y Github Community, donde se pueden ver preguntas similares que hayan hecho otros usuarios antes con los mismos problemas.

Finalmente se escogió una máquina EC2 de AWS (Amazon Web Services) en la que se subieron las tres partes de la aplicación web (proyecto Angular, proyecto .NET y base de datos) para ejecutarlas y dejar la máquina corriendo para mantener la aplicación en ejecución constantemente.

Esta subida también implicó modificar el código, ya que surgieron problemas nuevos porque las url de conexión del front-end con los endpoints generados por el back-end también cambiaron al no ser el localhost el que estaba en ejecución.

\imagen{urlNueva.png}{Cambio de las url de conexión con los endpoints}{0.9}

\section{Análisis del código con SonarQube}

Se ha analizado el código para ver si tiene problemas de seguridad, de fiabilidad, mantenibilidad y ver si hay código duplicado.

Como se puede ver en la imagen, el nivel de seguridad lo ha valorado como A, que es el mejor valor posible, el 0 indica que ha habido 0 vulnerabilidades. En la fiabilidad, tiene el nivel B, que indica que ha encontrado 6 vulnerabilidades con algunos posibles bugs que no afectan en el funcionamiento de la aplicación. En cuanto a la mantenibilidad, la nota también ha sido la mejor posible y indica que puede haber algunos problemas muy leves que no implican una corrección inmediata. El porcentaje de código duplicado también ha salido muy bajo (0.2), lo que indica que el código no se repite de forma innecesaria.

Por otro lado, se ha analizado también el código del back-end, pero en este caso, no hay una interfaz tan visual y es por comandos. La salida ha mostrado algunas advertencias sobre las versiones instaladas de los paquetes que se están utilizando en el proyecto, avisando de que tiene versiones más modernas y con menos problemas de seguridad.

\imagen{sonarqube.png}{Imágen del análisis del código en SonarQube}{0.9}